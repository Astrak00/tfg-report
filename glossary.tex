\makeglossaries
% \makenoidxglossaries  % slower, but no need to do $ makeglossaries report.tex


% usage:
% - \gls{term}: regular stuff
% - \Gls{term}: first uppercase
% - \glspl{term}: plural
% - \Glspl{term}: you know the drill
% - \glsdisp{term}{custom text}: link with custom text
% https://www.overleaf.com/learn/latex/Glossaries


\newglossaryentry{CPP} {
  name = {C++},
  description = {Compiled language, created by Bjarne Stroustrup in 1979}
}


\newglossaryentrywithacronym{RISC}
{Reduced Instruction Set Computer}
{A Reduced Instruction Set Computer is a type of microprocessor architecture that utilizes a small, highly-optimized set of instructions rather than the highly-specialized set of instructions typically found in other architectures.}


\newglossaryentry{CISC}{
    name = {Complex Instruction Set Computing},
    description = {}
}

\newglossaryentry{goroutine}{
    name = {goroutine},
    description = {A goroutine is a lightweight thread managed by the Go runtime}
}

\newglossaryentry{channel}{
    name = {channel},
    description = {
        A channel is a Go language construct that provides a way for goroutines to communicate with each other and synchronize their execution by sending and receiving values.
    }
}





\newacronym{jit}{JIT}{Just In Time Compiler}
\newacronym{SIMD}{SIMD}{Single Instruction Multiple Data}



\newglossaryentry{ai-accelerator}{
    name = {AI Accelerator},
    description = {An expansion card that usually has specific hardware to accelerate AI workloads}
}   

\newglossaryentry{cross-compilation}{
    name = {cross compilation},
    description = {Compiling a program with a different architecture than the host machine that is making the compilation.}
}

\newglossaryentrywithacronym{GIL}
{Global Interpreter Lock}
{The global interpreter lock is a mutex that protects access to Python objects, preventing multiple threads from executing Python bytecodes at once}

\newglossaryentry{CPython}{
    name = {CPython},
    description = {Pythons default interpreter, with a \gls{GIL}, until version 3.13 which an experimental free-threaded version was created and version 3.14 will continue its development (both known as 3.13t and 3.14t respectively)}
}

\newglossaryentrywithacronym{VM}
{Virtual Machine}
{The Python virtual machine is the same as the interpreter, which turns python's code into machine ready instructions. Pythons default VM implementation if \gls{CPython}, but others exists such as Jython (Java), IronPython (.NET) or PyPy}

\newglossaryentrywithacronym{GC}
{Garbage Collector}
{The garbage collector is an automatic memory management whose primary job is to reclaim memory that was allocated by the program but is no longer being used}

\newglossaryentry{heap-allocation}{
    name = {Heap Allocation},
    description = {Heap allocation is the process of reserving a block of memory from a large, flexible pool (the heap) for program data whose size or lifetime is not known at compile time, allowing for dynamic memory management during program execution. As python does not know the sizes of most variables, it needs to perform heap allocations always.}
}

\newglossaryentry{polymorphism}{
    name = {polymorphism},
    description = {allows objects of different types to be treated as objects of a common superclass or interface}
}

\newacronym{ai}{AI}{Artificial Intelligence}

\newacronym{PVM}{PVM}{Python's Virtual Machine}


\newglossaryentry{bytecode}{
    name = {bytecode},
    description = {Bytecode is a set of instructions for a hypothetical machine, for example the Python Virtual Machine (PVM). or the \href{https://www.java.com/en/}{Java} Virtual Machine (JVM)}
}

\newglossaryentrywithacronym{AST}
{Abstract Syntax Tree}
{An abstract syntax tree is a data structure used in computer science to represent the structure of a program or code snippet. It is a tree representation of the abstract syntactic structure of text (often source code) written in a formal language.}


\newacronym{cpu}{CPU}{Central Processing Unit}
\newacronym{gpu}{GPU}{Graphics Processing Unit}
\newacronym{isa}{ISA}{Instruction Set Architecture}
\newacronym{cisc}{CISC}{Complex Instruction Set Computer}
\newacronym{os}{OS}{Operating System}

\newglossaryentry{gc-cycle}{
    name = {Garbage Collection cycle},
    description = {A Garbage Collection Cycle is a single execution of the garbage collection system that identifies and reclaims memory from objects no longer in use by an application.}
}

\newglossaryentry{mutex}{
    name = {mutex},
    description = {Mutual Exclusion (mutex) is a synchronization primitive that ensures that only one thread can access a resource or critical section at a time, preventing race conditions and ensuring data integrity in concurrent programming.}
}

\newacronym{raii}{RAII}{Resource Acquisition Is Initialization}


\newglossaryentry{tri-color-mark-and-sweep}{
    name = {tri-color mark and sweep},
    description = {
        A garbage collection algorithm that uses three colors (white, gray, and black) to track the reachability of objects in memory. It marks reachable objects as gray, processes them to mark their references as black, and sweeps unmarked (white) objects for reclamation.
    }
}

\newacronym{i-o}{I/O}{Input / Output}
\newacronym{json}{JSON}{JavaScript Object Notation}
\newglossaryentry{c-extension}{
    name = {C extension},
    description = {
        In the Python standard library, C-extensions are modules written in C (and compiled into shared libraries) that expose high-performance, low-level functionality and native data types directly to the CPython interpreter.
    }
}

\newglossaryentry{LLVM}{
    name = {LLVM},
    description = {LLVM began as a research project at the University of Illinois, with the goal of providing a modern compilation strategy capable of supporting both static and dynamic compilation of arbitrary programming languages. The LLVM Project is a collection of modular and reusable compiler and toolchain technologies. Despite its name, LLVM has little to do with traditional virtual machines. The name "LLVM" itself is not an acronym; it is the full name of the project.}
}

\newglossaryentry{numpy}{
    name = {numpy},
    description = {The most popular python package for scientific computation with python}
}
\newglossaryentry{PyTorch}{
    name = {PyTorch},
    description = {An open-source machine learning framework for Python, providing GPU-accelerated tensor computation and automatic differentiation for building and training deep neural networks.}
}

\newglossaryentry{V8}{
    name = {V8},
    description = {Google's open-source, high-performance JavaScript and WebAssembly engine, written in C++, that powers the Google Chrome browser and the Node.js runtime environment}
}

\newacronym{ram}{RAM}{Random Acces Memory}
\newglossaryentry{fortran}{
    name = {fortran},
    description = {Fortran was a pioneering, high-level compiled programming language (short for Formula Translation) developed in the 1950s, which remains widely used for its high performance in scientific, engineering, and numerical computation.}
}

\newglossaryentrywithacronym{NUMA}
{Non-Uniform Memory Access}
{It's a computer memory design where the time it takes to access memory depends on which processor is accessing which memory location}

\newacronym{CCX}{CCX}{Core Complex}
\newacronym{fov}{FOV}{Field of Vision}

\newglossaryentrywithacronym{oop}
{Object-Oriented Programming}
{Programming paradigm that organizes code into reusable units called "objects," which contain both data and the functions that operate on that data}


\newglossaryentry{openMP}{
    name = {openMP},
    description = {OpenMP is an Application Program Interface (API) that may be used to explicitly direct multithreaded, shared memory parallelism in C/C++ programs}
}

\newglossaryentry{green-thread}{
    name = {Green threads},
    description = {Green threads are software threads managed by a runtime environment or library, rather than directly by the operating system}
}
\newglossaryentrywithacronym{wg}
{Wait Group}
{A Go WaitGroup is a synchronization primitive that uses an internal counter, incremented by Add when goroutines start and decremented by Done when they finish, to block Wait until all registered goroutines have completed.}

\newacronym{rgb}{RGB}{Red Green Blue}

\newglossaryentry{ppm}{
    name = {PPM},
    description = {A PPM (Portable PixMap) is an image file format in the Netpbm family that encodes uncompressed color images via a simple text-based header (width, height, max color value) followed by ASCII or binary \gls{rgb} pixel data.}
}

\newglossaryentry{DDR4}{
    name = {DDR4},
    description = {Double Data Rate 4 Memory is a type of volatyle memory standard in most computers. Data transfers occur on both the rising and falling edges of the system clock}
}

\newglossaryentry{htop}{
    name = {htop},
    description = {htop is an interactive process viewer for Unix systems, providing a real-time, user-friendly interface to monitor system processes, resource usage, and performance metrics. It is a more advanced alternative to the traditional top command.}
}

\newglossaryentry{sudo}{
    name = {sudo},
    description = {sudo is a command-line utility in Unix-like operating systems that allows a permitted user to execute a command as the superuser or another user, as specified by the security policy. It stands for "superuser do" and is commonly used to perform administrative tasks with elevated privileges.}
}

\newglossaryentry{NCSA}{
    name = {NCSA},
    description = {The NCSA (National Center for Supercomputing Applications) is a research center at the University of Illinois at Urbana-Champaign that played a significant role in the development of the web and high-performance computing.}
}

\newacronym{IDE}{IDE}{Integrated Development Environment}

\newglossaryentrywithacronym{fpga}
{Field Programmable Gate Array}
{A Field Programmable Gate Array is an integrated circuit that can be configured by the user after manufacturing to perform specific hardware tasks, allowing for customizable and reprogrammable logic functions.}