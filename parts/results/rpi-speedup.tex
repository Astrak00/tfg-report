\begin{figure}[]
  \centering

  % ----- first sub‑figure (relative efficiency) -----
  \begin{subfigure}[t]{0.48\textwidth}
    \centering
    \begin{tikzpicture}
      \begin{axis}[
          title={Relative Efficiency - RPI},
          width=0.5*\plotwidthgraph,
          height=0.8*\plotheightgraph,
          xlabel={Number of Cores},
          ylabel={Relative Efficiency},
          ymode=log,
          grid=both,
          minor tick num=1,
          grid style={gray!30,dashed},
          xtick={1,2,4},
          x tick label style={
            font=\footnotesize,
            rotate=45,
            anchor=north east
          },
          legend style={
            at={(0.3,0.65)},
            anchor=north west,
            font=\scriptsize,
            nodes={scale=0.8,transform shape},
            draw=none
          },
          legend columns=2,
          transpose legend,
          legend cell align=left,
        ]

        % --- data ---
        \addplot[blue,mark=o,line width=1pt] table[row sep=\\]{
          x   y\\
          1  100.00\\
          2   64.58\\
          4   51.60\\
        };
        \addlegendentry{C++}

        \addplot[violet,mark=o,line width=1pt] table[row sep=\\]{
          x   y\\
          1  295.83\\
          2  191.67\\
          4  134.00\\
        };
        \addlegendentry{Go}

        \addplot[orange,mark=o,line width=1pt] table[row sep=\\]{
          x   y\\
          1  237.50\\
          2  190.42\\
          4  227.00\\
        };
        \addlegendentry{PyPy 3.11.11}

        \addplot[green!60!black,mark=o,line width=1pt] table[row sep=\\]{
          x   y\\
          1 14725.00\\
          2  9525.83\\
          4  7173.75\\
        };
        \addlegendentry{Python}
      \end{axis}
    \end{tikzpicture}
    \caption[RPi - Relative Energy Efficiency]{Relative energy usage on the Raspberry, using C++ single core set as baseline set at 100 (lower is better).}
    \label{fig:rpi-relative-energy-pkg}
  \end{subfigure}
  \hfill
  % ----- second sub‑figure (speedup) -----
  \begin{subfigure}[t]{0.48\textwidth}
    \centering
    \begin{tikzpicture}
      \begin{axis}[
          title={Execution Time Speedup - RPI},
          width=0.5*\plotwidthgraph,
          height=0.8*\plotheightgraph,
          xlabel={Number of Cores},
          ylabel={Speedup},
          ymode=log,
          grid=both,
          minor tick num=1,
          grid style={gray!30,dashed},
          xtick={1,2,4},
          x tick label style={
            font=\footnotesize,
            rotate=45,
            anchor=north east
          },
          legend style={
            at={(0.3,0.65)},
            anchor=north west,
            font=\scriptsize,
            nodes={scale=0.8,transform shape},
            draw=none
          },
          legend columns=2,
          transpose legend,
          legend cell align=left,
        ]

        % --- data ---
        \addplot[blue,mark=o,line width=1pt] table[row sep=\\]{
          x   y\\
          1  100\\
          2  51.25897718\\
          4  25.98195835\\
        };
        \addlegendentry{C++}

        \addplot[violet,mark=o,line width=1pt] table[row sep=\\]{
          x   y\\
          1 334.9893837\\
          2 166.4016986\\
          4  83.43286551\\
        };
        \addlegendentry{Go}

        \addplot[orange,mark=o,line width=1pt] table[row sep=\\]{
          x   y\\
          1 256.5027464\\
          2 151.0628494\\
          4 146.428006\\
        };
        \addlegendentry{PyPy 3.11.11}

        \addplot[green!60!black,mark=o,line width=1pt] table[row sep=\\]{
          x   y\\
          1 17970.75486\\
          2 8980.04214\\
          4 4596.681539\\
        };
        \addlegendentry{Python}

      \end{axis}
    \end{tikzpicture}
    \caption[RPI - Execution Time Speedup]{Relative execution time on the Raspberry, using C++ single core set as baseline set at 100 (lower is better).}
    \label{fig:rpi-time-speedup}
  \end{subfigure}

  \caption{Raspberry Pi - Relative execution and energy consumption.}
  \label{fig:rpi-relative-combined}
\end{figure}
