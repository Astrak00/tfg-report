\begin{figure}
    \centering
    \begin{tikzpicture}
  \begin{axis}[
      title={Energy Efficiency Speedup - Raspberry Pi},
      width=\plotwidthgraph,
      height=\plotheightgraph,
      xlabel={Number of Cores},
      ylabel={Relative Efficiency},
      ymode=linear,
      xmode=linear,
      grid=both,
      minor tick num=1,
      grid style={gray!30,dashed},
      xtick={1,2,4},
      x tick label style={
        font=\footnotesize,
        rotate=45,
        anchor=north east
      },
      legend style={
        at={(0.02,0.98)},
        anchor=north west,
        font=\scriptsize,
        nodes={scale=0.8,transform shape},
        draw=none
      },
      legend columns=2,
      transpose legend,
      legend cell align=left,
    ]
    %% C++ %%
    \addplot[
      blue,
      mark=o,
    ]
    table[row sep=\\] {
      x   y     \\
      1   1 \\
      2   1.548387097 \\
      4   1.937984496 \\
    };
    \addlegendentry{C++}

    %% Go %%
    \addplot[
      violet,
      mark=o,
    ]
    table[row sep=\\] {
      x   y     \\
      1   1 \\
      2   1.543478261 \\
      4   2.207711443 \\
    };
    \addlegendentry{Go}

    %% PyPy 3.11.11 %%
    \addplot[
      orange,
      mark=o,
    ]
    table[row sep=\\] {
      x   y     \\
      1   1 \\
      2   1.24726477 \\
      4   1.046255507 \\
    };
    \addlegendentry{PyPy 3.11.11}

    %% Python %%
    \addplot[
      green!60!black,
      mark=o,
    ]
    table[row sep=\\] {
      x   y     \\
      1   1 \\
      2   1.545796518 \\
      4   2.052622408 \\
    };
    \addlegendentry{Python}

  \end{axis}
\end{tikzpicture}
    \caption{Raspberry Pi - Energy Efficiency Speedup}{Energy efficiency speedup of the Raspberry Pi benchmark across different programming languages.}
    \label{fig:rpi-speedup}
\end{figure}
