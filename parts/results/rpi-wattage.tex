\begin{figure}
 \centering
    \begin{tikzpicture}
    \begin{axis}[
    title={Wattage - Raspberry Pi},
    width=\plotwidthgraph,
    height=\plotheightgraph,
    xlabel={Number of Cores},
    ylabel={Wattage (W)},
    ymode=linear,
    xmode=linear,
    grid=both,
    minor tick num=1,
    grid style={gray!30,dashed},
    xtick={1,2,4},
    x tick label style={
    font=\footnotesize,
    rotate=45,
    anchor=north east
    },
    legend style={
    at={(0.02,0.95)},
    anchor=north west,
    font=\scriptsize,
    nodes={scale=0.8,transform shape},
    draw=none
    },
    legend columns=2,
    transpose legend,
    legend cell align=left,
    ]
    %% C++ %%
        \addplot[
        blue,
        mark=o,
        line width=0.75pt,
        ]
        table[row sep=\\] {
        x y \\
        1  1.62  \\
        2  2.05  \\
        4  3.23  \\
        };
        \addlegendentry{C++}
    %% Go %%
        \addplot[
        violet,
        mark=o,
        line width=0.75pt,
        ]
        table[row sep=\\] {
        x y \\
        1  1.43  \\
        2  1.87  \\
        4  2.61  \\
        };
        \addlegendentry{Go}
    %% PyPy 3.11.11 %%
        \addplot[
        orange,
        mark=o,
        line width=0.75pt,
        ]
        table[row sep=\\] {
        x y \\
        1  1.50   \\
        2  2.05   \\
        4  2.52   \\
        };
        \addlegendentry{PyPy 3.11.11}
    %% Python %%
        \addplot[
        green!60!black,
        mark=o,
        line width=0.75pt,
        ]
        table[row sep=\\] {
        x y \\
        1  1.33   \\
        2  1.72   \\
        4  2.53   \\
        };
        \addlegendentry{Python}
    \end{axis}
    \end{tikzpicture}
\caption[Raspberry Pi - Wattage]{Wattage of the Raspberry Pi benchmark across different programming languages.}
\label{fig:rpi-wattage}
\end{figure}
