\chapter{Problem Statement \& Analysis}\label{chap:analysis}

This chapter contains an analysis of the functionalities provided by the simulator and the accompanying helper files. The objective of this analysis is to provide a clear understanding of the requirements and functionalities that the simulator must fulfill, as well as the constraints and non-functional requirements that must be considered during its development.

To achieve this, we will define the user requirements, functional and non-functional requirements, and restrictions that the simulator must adhere to. Then, we will present a use case diagram that illustrates the interactions between the user and the simulator, as well as traceability matrices that link the requirements to the use cases and functionalities of the simulator.

\section{Project Description}
This project aims to create a suite of programs that implement a ray-tracer engine in multiple programming languages, including C++, Python and Go. The goal is creating the pipeline to benchmark the performance and energy efficiency of each implementation, in multiple-core and single-core configurations.

Each implementation has to be able to run on macOS and Linux, and the energy consumption evaluation of each benchmark must work on each platform, as not all platforms support the same energy consumption evaluation tools.\footnote{macOS uses \texttt{powermetrics} and Linux uses \texttt{perf} and a Raspberry Pi needs its power measured from the input source, as it does not have any internal counters} This project is only implemented to run on \glspl{cpu} as this projects' scope does not include \glspl{gpu}.


\section{Requirements}


The requirements for this project are divided into two main categories user requirements and program requirement. All of these requirements have been defined following the standard~\cite{requirements-engineering-iso}. 

To better organize the requirements, we will use the following abbreviations:

% Abbreviations used in this document:
\begin{description}
    \item[UR]  User Requirement
    \item[CA]  Capacity
    \item[RE]  Restriction
    \item[FN]  Functional
    \item[NF]  Non-Functional
\end{description}

\subsection{User Requirements}\label{sec:user-requirements}
This section describes the user requirements, those derived from the needs of the users of the end system. There are two main types of user requirements: capacity and restriction. The capacity requirements describe the functionalities that the user expects from the system, while the restriction requirements describe the constraints that the system must adhere to, defining the limitations of the system.

To better describe these requirements, we will use the following format:

\begin{table}[H]
    \centering
    \begin{tabular}{l p{10cm}}
        \toprule
        \multicolumn{2}{c}{UR-CA-XX} \\
        \toprule
        \textbf{Description}        & \textit{Requirement's Description} \\
        \textbf{Necessity}          & \textit{Low / Medium / High} \\
        \textbf{Priority}           & \textit{Low / Medium / High} \\
        \multirow{1}{*}{\textbf{Stability}} & \textit{Stable / Unstable}: How easy it is for the requirement to change \\
                                            & along the development of the project \\
        \textbf{Verifiability}       & \textit{Verifiable / Non-Verifiable} \\
    \end{tabular}
    \caption{Requirement UR-CA-XX}
\end{table}

The ID of the requirement is composed of the prefix \textit{UR} (User Requirement), followed by a dash, then the type of requirement (\textit{CA} for Capacity, \textit{RE} for Restriction), and finally a two-digit number that identifies the requirement.

\subsubsection{Capacity}

\begin{table}[H]
    \centering
    \begin{tabular}{l p{10cm}}
        \toprule
        \multicolumn{2}{c}{UR-CA-01} \\
        \toprule
        \textbf{Description}        &  The user must be able to run the language benchmarks on macOS and Linux.  \\
        \textbf{Necessity}          &  High   \\
        \textbf{Priority}           &  Medium   \\
        \textbf{Stability}          &  Stable   \\
        \textbf{Verifiability}       &  Verifiable \\
    \end{tabular}
\caption{Requirement UR-CA-01}\label{tab:ur-ca-01}
\end{table}

\begin{table}[H]
    \centering
    \begin{tabular}{l p{10cm}}
        \toprule
        \multicolumn{2}{c}{UR-CA-02} \\
        \toprule
        \textbf{Description}        & The user must be able to inspect every aspect of the code. \\
        \textbf{Necessity}          & High  \\
        \textbf{Priority}           & High  \\
        \textbf{Stability}          & Stable \\
        \textbf{Verifiability}       & Verifiable \\
    \end{tabular}
    \caption{Requirement UR-CA-02}\label{tab:ur-ca-02}
\end{table}

\begin{table}[H]
    \centering
    \begin{tabular}{l p{10cm}}
        \toprule
        \multicolumn{2}{c}{UR-CA-03} \\
        \toprule
        \textbf{Description}        & The user must be able to add their own implementation of the program to be tested. \\
        \textbf{Necessity}          &  High   \\
        \textbf{Priority}           &  High   \\
        \textbf{Stability}          &  Stable \\
        \textbf{Verifiability}       &  Verifiable \\
    \end{tabular}
    \caption{Requirement UR-CA-03}\label{tab:ur-ca-03}
\end{table}

\begin{table}[H]
    \centering
    \begin{tabular}{l p{10cm}}
        \toprule
        \multicolumn{2}{c}{UR-CA-04} \\
        \toprule
        \textbf{Description}        & The user must be able to check the outputs of the programs. \\
        \textbf{Necessity}          & High   \\
        \textbf{Priority}           & High   \\
        \textbf{Stability}          & Stable \\
        \textbf{Verifiability}      & Verifiable \\
    \end{tabular}
    \caption{Requirement UR-CA-04}\label{tab:ur-ca-04}
\end{table}

\begin{table}[H]
    \centering
    \begin{tabular}{l p{10cm}}
        \toprule
        \multicolumn{2}{c}{UR-CA-05} \\
        \toprule
        \textbf{Description}        & The user must be able to check the energy consumption of the benchmarks per-core\\
        \textbf{Necessity}          & High   \\
        \textbf{Priority}           & High   \\
        \textbf{Stability}          & Stable \\
        \textbf{Verifiability}      & Verifiable \\
    \end{tabular}
    \caption{Requirement UR-CA-05}\label{tab:ur-ca-05}
\end{table}

\begin{table}[H]
    \centering
    \begin{tabular}{l p{10cm}}
        \toprule
        \multicolumn{2}{c}{UR-CA-06} \\
        \toprule
        \textbf{Description}        & The user must be able to decide the amount of cores needed via the CLI\\
        \textbf{Necessity}          & High   \\
        \textbf{Priority}           & High   \\
        \textbf{Stability}          & Stable \\
        \textbf{Verifiability}      & Verifiable \\
    \end{tabular}
    \caption{Requirement UR-CA-06}\label{tab:ur-ca-06}
\end{table}

\begin{table}[H]
    \centering
    \begin{tabular}{l p{10cm}}
        \toprule
        \multicolumn{2}{c}{UR-CA-07} \\
        \toprule
        \textbf{Description}        & The user must be able to specify if the benchmark should use \texttt{taskset} to fix the number of available cores.\\
        \textbf{Necessity}          & High   \\
        \textbf{Priority}           & High   \\
        \textbf{Stability}          & Stable \\
        \textbf{Verifiability}      & Verifiable \\
    \end{tabular}
    \caption{Requirement UR-CA-07}\label{tab:ur-ca-07}
\end{table}

\begin{table}[H]
    \centering
    \begin{tabular}{l p{10cm}}
        \toprule
        \multicolumn{2}{c}{UR-CA-08} \\
        \toprule
        \textbf{Description}        & The user should be able to modify the execution command to add specific performance metric and energy consumption parameters before the execution of the benchmark. \\
        \textbf{Necessity}          & Low   \\
        \textbf{Priority}           & Low   \\
        \textbf{Stability}          & Unstable \\
        \textbf{Verifiability}      & Verifiable \\
    \end{tabular}
    \caption{Requirement UR-CA-08}\label{tab:ur-ca-08}
\end{table}

\begin{table}[H]
    \centering
    \begin{tabular}{l p{10cm}}
        \toprule
        \multicolumn{2}{c}{UR-CA-09} \\
        \toprule
        \textbf{Description}        & The user must be able to check the time taken to run the different programs / benchmarks \\
        \textbf{Necessity}          & High   \\
        \textbf{Priority}           & High   \\
        \textbf{Stability}          & Stable \\
        \textbf{Verifiability}      & Verifiable \\
    \end{tabular}
    \caption{Requirement UR-CA-09}\label{tab:ur-ca-09}
\end{table}

\begin{table}[H]
    \centering
    \begin{tabular}{l p{10cm}}
        \toprule
        \multicolumn{2}{c}{UR-CA-10} \\
        \toprule
        \textbf{Description}        & The user must be able to change the parameters of the image generated \\
        \textbf{Necessity}          & High   \\
        \textbf{Priority}           & High   \\
        \textbf{Stability}          & Stable \\
        \textbf{Verifiability}      & Verifiable \\
    \end{tabular}
    \caption{Requirement UR-CA-10}\label{tab:ur-ca-10}
\end{table}

\begin{table}[H]
    \centering
    \begin{tabular}{l p{10cm}}
        \toprule
        \multicolumn{2}{c}{UR-CA-11} \\
        \toprule
        \textbf{Description}        & The user must be able to visualize the resulting images of the program execution \\
        \textbf{Necessity}          & High   \\
        \textbf{Priority}           & High   \\
        \textbf{Stability}          & Stable \\
        \textbf{Verifiability}      & Verifiable \\
    \end{tabular}
    \caption{Requirement UR-CA-11}\label{tab:ur-ca-11}
\end{table}

\begin{table}[H]
    \centering
    \begin{tabular}{l p{10cm}}
        \toprule
        \multicolumn{2}{c}{UR-CA-12} \\
        \toprule
        \textbf{Description}        & The user must be able to compare the resulting images of the program execution \\
        \textbf{Necessity}          & High   \\
        \textbf{Priority}           & High   \\
        \textbf{Stability}          & Stable \\
        \textbf{Verifiability}      & Verifiable \\
    \end{tabular}
    \caption{Requirement UR-CA-12}\label{tab:ur-ca-12}
\end{table}

\begin{table}[H]
    \centering
    \begin{tabular}{l p{10cm}}
        \toprule
        \multicolumn{2}{c}{UR-CA-13} \\
        \toprule
        \textbf{Description}        & The system must inform the user how many cores it is using though the terminal \\
        \textbf{Necessity}          & Medium   \\
        \textbf{Priority}           & Low   \\
        \textbf{Stability}          & Stable \\
        \textbf{Verifiability}      & Verifiable \\
    \end{tabular}
    \caption{Requirement UR-CA-13}\label{tab:ur-ca-13}
\end{table}

\begin{table}[H]
    \centering
    \begin{tabular}{l p{10cm}}
        \toprule
        \multicolumn{2}{c}{UR-CA-14} \\
        \toprule
        \textbf{Description}        & The user must be able to compare different executions on the same platform. \\
        \textbf{Necessity}          & High   \\
        \textbf{Priority}           & High   \\
        \textbf{Stability}          & Stable \\
        \textbf{Verifiability}      & Verifiable \\
    \end{tabular}
    \caption{Requirement UR-CA-14}\label{tab:ur-ca-14}
\end{table}

\begin{table}[H]
    \centering
    \begin{tabular}{l p{10cm}}
        \toprule
        \multicolumn{2}{c}{UR-CA-15} \\
        \toprule
        \textbf{Description}        & The user must be able to view the remaining lines of the image to be renders. \\
        \textbf{Necessity}          & Low   \\
        \textbf{Priority}           & Low   \\
        \textbf{Stability}          & Stable \\
        \textbf{Verifiability}      & Verifiable \\
    \end{tabular}
    \caption{Requirement UR-CA-15}\label{tab:ur-ca-15}
\end{table}

\begin{table}[H]
    \centering
    \begin{tabular}{l p{10cm}}
        \toprule
        \multicolumn{2}{c}{UR-CA-16} \\
        \toprule
        \textbf{Description}        & The user must be able to run each benchmark individually. \\
        \textbf{Necessity}          & Medium   \\
        \textbf{Priority}           & Low   \\
        \textbf{Stability}          & Stable \\
        \textbf{Verifiability}      & Verifiable \\
    \end{tabular}
    \caption{Requirement UR-CA-16}\label{tab:ur-ca-16}
\end{table}

\begin{table}[H]
    \centering
    \begin{tabular}{l p{10cm}}
        \toprule
        \multicolumn{2}{c}{UR-CA-17} \\
        \toprule
        \textbf{Description}        & The user must be able to run all benchmarks of one language together. \\
        \textbf{Necessity}          & Medium   \\
        \textbf{Priority}           & Low   \\
        \textbf{Stability}          & Stable \\
        \textbf{Verifiability}      & Verifiable \\
    \end{tabular}
    \caption{Requirement UR-CA-17}\label{tab:ur-ca-17}
\end{table}

\subsubsection{Restriction}

\begin{table}[H]
    \centering
    \begin{tabular}{l p{10cm}}
        \toprule
        \multicolumn{2}{c}{UR-RE-01} \\
        \toprule
        \textbf{Description}        &  The same architecture should be used for every program \\
        \textbf{Necessity}          &  Low \\
        \textbf{Priority}           &  Low \\
        \textbf{Stability}          &  Unstable \\
        \textbf{Verifiability}      &  Verifiable \\
    \end{tabular}
    \caption{Requirement UR-RE-01}\label{tab:ur-re-01}
\end{table}

\begin{table}[H]
    \centering
    \begin{tabular}{l p{10cm}}
        \toprule
        \multicolumn{2}{c}{UR-RE-02} \\
        \toprule
        \textbf{Description}        & The system must have a `one-command' execution for any given language \\
        \textbf{Necessity}          & Medium   \\
        \textbf{Priority}           & Low   \\
        \textbf{Stability}          & Stable \\
        \textbf{Verifiability}      & Verifiable \\
    \end{tabular}
    \caption{Requirement UR-RE-02}\label{tab:ur-re-02}
\end{table}


\section{System requirements}

This section describes the system requirements, which are the requirements that the system must fulfill to be considered complete. These requirements are divided into two main categories: non-functional and functional requirements.

To better describe these requirements, we will use the following format:

\begin{table}
    \centering
    \begin{tabular}{l p{10cm}}
        \toprule
        \multicolumn{2}{c}{SR-ZZ-XX} \\
        \toprule
        \textbf{Description}        & A brief description of the requirement \\
        \textbf{Necessity}          & How important is this requirement (Low / Medium / High) \\
        \textbf{Priority}           & How quick this requirement needs to be implemented (Low / Medium / High) \\
        \textbf{Stability}          & What is the probability of this requirement changing (Stable / Unstable) \\
        \textbf{Verifiability}      & How easy is it to verify this requirement is implemented (Verifiable / Non-Verifiable) \\
        \textbf{Origin}             & References the user (\autoref{sec:user-requirements}) requirements that inspired this requirement.  \\
    \end{tabular}
    \caption{Requirement SR-ZZ-XX}
\end{table}

The ID of the requirement is composed of the prefix \textit{SR} (System Requirement), followed by a dash, then the type of requirement (\textit{NF} for Non-Functional, \textit{FN} for Functional), and finally a two-digit number that identifies the requirement.

\subsection{Non-Functional Requirements}
\begin{table}[H]
    \centering
    \begin{tabular}{l p{10cm}}
        \toprule
        \multicolumn{2}{c}{SR-NF-01} \\
        \toprule
        \textbf{Description}        & The different programs must have the same architecture. \\
        \textbf{Necessity}          &  High \\
        \textbf{Priority}           &  High \\
        \textbf{Stability}          &  Stable \\
        \textbf{Verifiability}      & Verifiable \\
        \textbf{Origin}             &  \textit{\nameref{tab:ur-re-01}}, \textit{\nameref{tab:ur-ca-14}} \\
    \end{tabular}
    \caption{Requirement SR-NF-01}\label{tab:sr-nf-01}
\end{table}

\begin{table}[H]
    \centering
    \begin{tabular}{l p{10cm}}
        \toprule
        \multicolumn{2}{c}{SR-NF-02} \\
        \toprule
        \textbf{Description}        & The system should be open-source \\
        \textbf{Necessity}          &  High \\
        \textbf{Priority}           &  High \\
        \textbf{Stability}          &  Stable \\
        \textbf{Verifiability}      & Verifiable \\
        \textbf{Origin}             &  \textit{\nameref{tab:ur-re-02}}, \textit{\nameref{tab:ur-ca-03}}, \textit{\nameref{tab:ur-ca-05}}, \textit{\nameref{tab:ur-ca-06}}, \textit{\nameref{tab:ur-ca-07}}, \textit{\nameref{tab:ur-ca-08}}, \textit{\nameref{tab:ur-ca-10}}, \textit{\nameref{tab:ur-ca-12}}, \textit{\nameref{tab:ur-ca-13}} \\
    \end{tabular}
    \caption{Requirement SR-NF-02}\label{tab:sr-nf-02}
\end{table}

\begin{table}[H]
    \centering
    \begin{tabular}{l p{10cm}}
        \toprule
        \multicolumn{2}{c}{SR-NF-03} \\
        \toprule
        \textbf{Description}        & The ray-tracer should be implemented in \gls{CPP} \\
        \textbf{Necessity}          &  High \\
        \textbf{Priority}           &  High \\
        \textbf{Stability}          &  Stable \\
        \textbf{Verifiability}      & Verifiable \\
        \textbf{Origin}             &  \textit{\nameref{tab:ur-re-01}} \\
    \end{tabular}
    \caption{Requirement SR-NF-03}\label{tab:sr-nf-03}
\end{table}

\begin{table}[H]
    \centering
    \begin{tabular}{l p{10cm}}
        \toprule
        \multicolumn{2}{c}{SR-NF-04} \\
        \toprule
        \textbf{Description}        & The ray-tracer should be implemented in Python \\
        \textbf{Necessity}          &  High \\
        \textbf{Priority}           &  High \\
        \textbf{Stability}          &  Stable \\
        \textbf{Verifiability}      & Verifiable \\
        \textbf{Origin}             &  \textit{\nameref{tab:ur-re-01}} \\
    \end{tabular}
    \caption{Requirement SR-NF-04}\label{tab:sr-nf-04}
\end{table}

\begin{table}[H]
    \centering
    \begin{tabular}{l p{10cm}}
        \toprule
        \multicolumn{2}{c}{SR-NF-05} \\
        \toprule
        \textbf{Description}        & The ray-tracer should be implemented in Go \\
        \textbf{Necessity}          &  High \\
        \textbf{Priority}           &  High \\
        \textbf{Stability}          &  Stable \\
        \textbf{Verifiability}      & Verifiable \\
        \textbf{Origin}             &  \textit{\nameref{tab:ur-re-01}} \\
    \end{tabular}
    \caption{Requirement SR-NF-05}\label{tab:sr-nf-05}
\end{table}

\begin{table}[H]
    \centering
    \begin{tabular}{l p{10cm}}
        \toprule
        \multicolumn{2}{c}{SR-NF-06} \\
        \toprule
        \textbf{Description}        & The ray-tracer's output should be stored in a file. \\
        \textbf{Necessity}          &  High \\
        \textbf{Priority}           &  High \\
        \textbf{Stability}          &  Stable \\
        \textbf{Verifiability}      & Verifiable \\
        \textbf{Origin}             &  \textit{\nameref{tab:ur-ca-04}}, \textit{\nameref{tab:ur-ca-14}} \\
    \end{tabular}
    \caption{Requirement SR-NF-06}\label{tab:sr-nf-06}
\end{table}

\begin{table}[H]
    \centering
    \begin{tabular}{l p{10cm}}
        \toprule
        \multicolumn{2}{c}{SR-NF-07} \\
        \toprule
        \textbf{Description}        & The ray-tracer's randomness should not impact the structure of the scene \\
        \textbf{Necessity}          &  High \\
        \textbf{Priority}           &  High \\
        \textbf{Stability}          &  Stable \\
        \textbf{Verifiability}      & Verifiable \\
        \textbf{Origin}             & \textit{\nameref{tab:ur-ca-10}}, \textit{\nameref{tab:ur-ca-14}} \\
    \end{tabular}
    \caption{Requirement SR-NF-07}\label{tab:sr-nf-07}
\end{table}

\begin{table}[H]
    \centering
    \begin{tabular}{l p{10cm}}
        \toprule
        \multicolumn{2}{c}{SR-NF-08} \\
        \toprule
        \textbf{Description}        &  The system must implement different time-measuring systems for the different platforms. \\
        \textbf{Necessity}          &  High \\
        \textbf{Priority}           &  High \\
        \textbf{Stability}          &  Stable \\
        \textbf{Verifiability}      & Verifiable \\
        \textbf{Origin}             & \textit{\nameref{tab:ur-ca-09}}, \textit{\nameref{tab:ur-ca-14}} \\
    \end{tabular}
    \caption{Requirement SR-NF-08}\label{tab:sr-nf-08}
\end{table}

\subsection{Functional Requirements}

\begin{table}[H]
    \centering
    \begin{tabular}{l p{10cm}}
        \toprule
        \multicolumn{2}{c}{SR-FN-01} \\
        \toprule
        \textbf{Description}        &  The system must measure the energy consumption of each of the benchmarks regardless of the implementation language. \\
        \textbf{Necessity}          &  High \\
        \textbf{Priority}           &  High \\
        \textbf{Stability}          &  Stable \\
        \textbf{Verifiability}      & Verifiable \\
        \textbf{Origin}             & \textit{\nameref{tab:ur-ca-05}}, \textit{\nameref{tab:ur-ca-09}}, \textit{\nameref{tab:ur-ca-16}}, \textit{\nameref{tab:ur-ca-17}} \\
    \end{tabular}
    \caption{Requirement SR-FN-01}\label{tab:sr-fn-01}
\end{table}

\begin{table}[H]
    \centering
    \begin{tabular}{l p{10cm}}
        \toprule
        \multicolumn{2}{c}{SR-FN-02} \\
        \toprule
        \textbf{Description}        &  The system must output the file where the energy consumption result is kept at the end of the execution. \\
        \textbf{Necessity}          &  High \\
        \textbf{Priority}           &  High \\
        \textbf{Stability}          &  Stable \\
        \textbf{Verifiability}      & Verifiable \\
        \textbf{Origin}             & \textit{\nameref{tab:ur-ca-09}}, \textit{\nameref{tab:ur-ca-14}} \\
    \end{tabular}
    \caption{Requirement SR-FN-02}\label{tab:sr-fn-02}
\end{table}

\begin{table}[H]
    \centering
    \begin{tabular}{l p{10cm}}
        \toprule
        \multicolumn{2}{c}{SR-FN-03} \\
        \toprule
        \textbf{Description}        &  The system must output the file where the execution time is kept at the end of the execution. \\
        \textbf{Necessity}          &  High \\
        \textbf{Priority}           &  High \\
        \textbf{Stability}          &  Stable \\
        \textbf{Verifiability}      & Verifiable \\
        \textbf{Origin}             & \textit{\nameref{tab:ur-ca-09}}, \textit{\nameref{tab:ur-ca-14}} \\
    \end{tabular}
    \caption{Requirement SR-FN-03}\label{tab:sr-fn-03}
\end{table}

\begin{table}[H]
    \centering
    \begin{tabular}{l p{10cm}}
        \toprule
        \multicolumn{2}{c}{SR-FN-04} \\
        \toprule
        \textbf{Description}        &  The system must reduce outlier results by executing the benchmarks multiple times. \\
        \textbf{Necessity}          &  High \\
        \textbf{Priority}           &  High \\
        \textbf{Stability}          &  Stable \\
        \textbf{Verifiability}      & Verifiable \\
        \textbf{Origin}             & \textit{\nameref{tab:ur-ca-12}}, \textit{\nameref{tab:ur-ca-16}}, \textit{\nameref{tab:ur-ca-17}} \\
    \end{tabular}
    \caption{Requirement SR-FN-04}\label{tab:sr-fn-04}
\end{table}

\begin{table}[H]
    \centering
    \begin{tabular}{l p{10cm}}
        \toprule
        \multicolumn{2}{c}{SR-FN-05} \\
        \toprule
        \textbf{Description}        &  The system must show the lines that are remaining to be processed. \\
        \textbf{Necessity}          &  High \\
        \textbf{Priority}           &  High \\
        \textbf{Stability}          &  Stable \\
        \textbf{Verifiability}      & Verifiable \\
        \textbf{Origin}             & \textit{\nameref{tab:ur-ca-15}} \\
    \end{tabular}
    \caption{Requirement SR-FN-05}\label{tab:sr-fn-05}
\end{table}

\begin{table}[H]
    \centering
    \begin{tabular}{l p{10cm}}
        \toprule
        \multicolumn{2}{c}{SR-FN-06} \\
        \toprule
        \textbf{Description}        &  The system must have a script to process macOS Power metrics results. \\
        \textbf{Necessity}          &  High \\
        \textbf{Priority}           &  High \\
        \textbf{Stability}          &  Stable \\
        \textbf{Verifiability}      & Verifiable \\
        \textbf{Origin}             & \textit{\nameref{tab:ur-ca-05}} \\
    \end{tabular}
    \caption{Requirement SR-FN-06}\label{tab:sr-fn-06}
\end{table}

\begin{table}[H]
    \centering
    \begin{tabular}{l p{10cm}}
        \toprule
        \multicolumn{2}{c}{SR-FN-07} \\
        \toprule
        \textbf{Description}        &  The system must have a file from which all benchmarks can be launched. \\
        \textbf{Necessity}          &  High \\
        \textbf{Priority}           &  High \\
        \textbf{Stability}          &  Stable \\
        \textbf{Verifiability}      & Verifiable \\
        \textbf{Origin}             & \textit{\nameref{tab:ur-ca-05}}, \textit{\nameref{tab:ur-ca-14}}, \textit{\nameref{tab:ur-ca-17}} \\
    \end{tabular}
    \caption{Requirement SR-FN-07}\label{tab:sr-fn-07}
\end{table}

\begin{table}[H]
    \centering
    \begin{tabular}{l p{10cm}}
        \toprule
        \multicolumn{2}{c}{SR-FN-08} \\
        \toprule
        \textbf{Description}        &  The system must have a file from which parameters for the simulation can be changed. \\
        \textbf{Necessity}          &  High \\
        \textbf{Priority}           &  High \\
        \textbf{Stability}          &  Stable \\
        \textbf{Verifiability}      & Verifiable \\
        \textbf{Origin}             & \textit{\nameref{tab:ur-ca-05}}, \textit{\nameref{tab:ur-ca-06}}, \textit{\nameref{tab:ur-ca-14}} \\
    \end{tabular}
    \caption{Requirement SR-FN-08}\label{tab:sr-fn-08}
\end{table}

\begin{table}[H]
    \centering
    \begin{tabular}{l p{10cm}}
        \toprule
        \multicolumn{2}{c}{SR-FN-09} \\
        \toprule
        \textbf{Description}        &  The system must have a utility that checks the difference between two output images. \\
        \textbf{Necessity}          &  High \\
        \textbf{Priority}           &  High \\
        \textbf{Stability}          &  Stable \\
        \textbf{Verifiability}      & Verifiable \\
        \textbf{Origin}             & \textit{\nameref{tab:ur-ca-04}} \\
    \end{tabular}
    \caption{Requirement SR-FN-09}\label{tab:sr-fn-09}
\end{table}


\begin{table}[H]
    \centering
    \begin{tabular}{l p{10cm}}
        \toprule
        \multicolumn{2}{c}{SR-FN-10} \\
        \toprule
        \textbf{Description}        &  The running of the benchmarks on any device must not interfere with the reading of the energy consumed by the program. \\
        \textbf{Necessity}          &  High \\
        \textbf{Priority}           &  High \\
        \textbf{Stability}          &  Stable \\
        \textbf{Verifiability}      & Verifiable \\
        \textbf{Origin}             & \textit{\nameref{tab:ur-ca-09}} \\
    \end{tabular}
    \caption{Requirement SR-FN-10}\label{tab:sr-fn-10}
\end{table}

\begin{table}[H]
    \centering
    \begin{tabular}{l p{10cm}}
        \toprule
        \multicolumn{2}{c}{SR-FN-11} \\
        \toprule
        \textbf{Description}        &  The running of the benchmarks on any device must not interfere with the reading of the time taken to execute the program. \\
        \textbf{Necessity}          &  High \\
        \textbf{Priority}           &  High \\
        \textbf{Stability}          &  Stable \\
        \textbf{Verifiability}      & Verifiable \\
        \textbf{Origin}             & \textit{\nameref{tab:ur-ca-09}} \\
    \end{tabular}
    \caption{Requirement SR-FN-11}\label{tab:sr-fn-11}
\end{table}

\section{Use Case}\label{sec:use-case}

This section describes the use cases of the system. The use cases are represented in a diagram that shows the interactions between the user and the system. 

We have many use cases, each representing a specific functionality of the system. As this is a discrete system, we have one use case per benchmark that can be run. This can be seen in \autoref{fig:use-case-v2}

\begin{figure}
    \hspace*{-2cm}
    \begin{tikzpicture}
        \begin{umlsystem}[] {Use Case Diagram for the Ray-Tracer Benchmarker} 
            \umlusecase[name=cpp,x=4,y=-2,width=2.2cm,fill=blue!20] {Run all C++ benchmarks}
            \umlusecase[name=cpp-sc,x=10,y=2,width=2.2cm,fill=blue!20] {Run C++ single-core}
            \umlusecase[name=cpp-mc,x=10,y=0,width=2.2cm,fill=blue!20] {Run C++ multi-core}
            \umlusecase[name=go,x=4,y=-4,width=2.2cm,fill=violet!20] {Run all Go benchmarks}
            \umlusecase[name=go-sc,x=10,y=-2,width=2.2cm,fill=violet!20] {Run Go single-core}
            \umlusecase[name=go-mc,x=10,y=-4,width=2.2cm,fill=violet!20] {Run Go multi-core}
            \umlusecase[name=python,x=4,y=-6,width=2.2cm,fill=orange!20] {Run all Python benchmarks}
            \umlusecase[name=python-sc,x=10,y=-7,width=2.2cm,fill=orange!20] {Run Python single-core}
            \umlusecase[name=python-mc,x=10,y=-9,width=2.2cm,fill=orange!20] {Run Python multi-core}
            \umlusecase[name=pypy,x=4,y=-8,width=2.2cm,fill=green!20] {Run all PyPy benchmarks}
            \umlusecase[name=pypy-sc,x=10,y=-11,width=2.2cm,fill=green!20] {Run PyPy single-core}
            \umlusecase[name=pypy-mc,x=10,y=-13,width=2.2cm,fill=green!20] {Run PyPy multi-core}
            
            \umlusecase[name=all,y=-5,width=2.2cm] {All benchmarks}
            \umlusecase[name=all-sc,x=12,y=5.5,width=2.2cm] {All single-core benchmarks}
            \umlusecase[name=all-mc,x=12,y=-16.5,width=2.2cm] {All multicore benchmarks}

            \umlusecase[name=view-image,y=-10,width=2.2cm] {View image}

            

        \end{umlsystem}
        \node [above] at (current bounding box.north) {Ray-Tracer Benchmarker};

        \umlactor[y=-5,x=-3.2] {User}

        \umlassoc{User}{cpp}
        \umlassoc{User}{cpp-sc}
        \umlassoc{User}{cpp-mc}
        \umlassoc{User}{go}
        \umlassoc{User}{go-sc}
        \umlassoc{User}{go-mc}
        \umlassoc{User}{python}
        \umlassoc{User}{python-sc}
        \umlassoc{User}{python-mc}
        \umlassoc{User}{pypy}
        \umlassoc{User}{pypy-sc}
        \umlassoc{User}{pypy-mc}
        \umlassoc{User}{all}
        \umlassoc{User}{all-sc}
        \umlassoc{User}{all-mc}
        \umlassoc{User}{view-image}



        % CPP extends single and multi
        \draw [tikzuml dependency style] (cpp) -- node[above] {\scriptsize $\ll \text{extend} \gg$} (cpp-sc);
        \draw [tikzuml dependency style] (cpp) -- node[above] {\scriptsize $\ll \text{extend} \gg$} (cpp-mc);
        % Go 
        \draw [tikzuml dependency style] (go) -- node[below] {\scriptsize $\ll \text{extend} \gg$} (go-sc);
        \draw [tikzuml dependency style] (go) -- node[above] {\scriptsize $\ll \text{extend} \gg$} (go-mc);
        % Python
        \draw [tikzuml dependency style] (python) -- node[above] {\scriptsize $\ll \text{extend} \gg$} (python-sc);
        \draw [tikzuml dependency style] (python) -- node[above] {\scriptsize $\ll \text{extend} \gg$} (python-mc);
        % PyPy
        \draw [tikzuml dependency style] (pypy) -- node[above] {\scriptsize $\ll \text{extend} \gg$} (pypy-sc);
        \draw [tikzuml dependency style] (pypy) -- node[below] {\scriptsize $\ll \text{extend} \gg$} (pypy-mc);
        % PyPy
        \draw [tikzuml dependency style] (all) to[bend right=20] node[midway, below] {\scriptsize $\ll \text{extend} \gg$} (all-mc);
        \draw [tikzuml dependency style] (all) to[bend left=20] node[midway, above] {\scriptsize $\ll \text{extend} \gg$} (all-sc);


        % Single
        \draw [tikzuml dependency style] (all-sc) to[bend left=0] node[midway, above] { \scriptsize $\ll \text{extend} \gg$} (cpp-sc);
        \draw [tikzuml dependency style] (all-sc) to[bend left=35] node[midway, above] {\scriptsize $\ll \text{extend} \gg$} (go-sc);
        \draw [tikzuml dependency style] (all-sc) to[bend left=26] node[midway, above] {\scriptsize $\ll \text{extend} \gg$} (python-sc);
        \draw [tikzuml dependency style] (all-sc) to[bend left=38] node[midway, above] {\scriptsize $\ll \text{extend} \gg$} (pypy-sc);

        % Multi-core
        \draw [tikzuml dependency style] (all-mc) to[bend right=38] node[midway, above] {\scriptsize $\ll \text{extend} \gg$} (cpp-mc);
        \draw [tikzuml dependency style] (all-mc) to[bend right=26] node[midway, above] {\scriptsize $\ll \text{extend} \gg$} (go-mc);
        \draw [tikzuml dependency style] (all-mc) to[bend right=35] node[midway, above] {\scriptsize $\ll \text{extend} \gg$} (python-mc);
        \draw [tikzuml dependency style] (all-mc) to[bend right=0] node[midway, above] {\scriptsize $\ll \text{extend} \gg$} (pypy-mc);
        
    \end{tikzpicture}
    \caption{Use Case Diagram for the Ray-Tracer Benchmarker}
    \label{fig:use-case}
\end{figure}


\newpage

\begin{landscape}
\begin{figure}
    \hspace*{-1.5cm}
    \centering
    \begin{tikzpicture}
        \umlactor[name=user,x=-4,y=-14,] {User}

        \begin{umlsystem}[x=-2,y=-12]{Ray-Tracer Benchmarker}

            \umlusecase[x =  0, y =  0, name=runall]{UC-01: Run all benchmarks}
            \umlusecase[x =  6, y =  0, width=2cm, name=runcpp,    fill=blue!20]{UC-02: Run C++ benchmarks}
            \umlusecase[x =  6, y = -3, width=2cm, name=rungo,     fill=violet!20]{UC-03: Run Go benchmarks}
            \umlusecase[x =  6, y = -6, width=2cm, name=runpy,     fill=orange!20]{UC-04: Run Python benchmarks}
            \umlusecase[x =  6, y = -9, width=2cm, name=runpypy,   fill=green!20]{UC-05: Run PyPy benchmarks}

            \umlusecase[x = 14, y =  0, width = 2.5cm, fill = gray!10, name=single]{UC-06:\\Run single-core}
            \umlusecase[x = 14, y = -3, width = 2.5cm, fill = gray!10, name=multi]{UC-07:\\Run multicore}

            \umlusecase[width=3cm, x = 20, y =  2, name=cppsingle,fill=blue!20]{UC-02.1: C++ single-core}
            \umlusecase[width=3cm, x = 20, y = 0.25, name=cppmulti,fill=blue!20]{UC-02.2: C++ multi-core}
            \umlusecase[width=3cm, x = 20, y = -1.5, name=gosingle,fill=violet!20]{UC-03.1: Go single-core}
            \umlusecase[width=3cm, x = 20, y = -3.25, name=gomulti,fill=violet!20]{UC-03.2: Go multi-core}
            \umlusecase[width=3cm, x = 20, y = -5, name=pysingle,fill=orange!20]{UC-04.1: Python single-core}
            \umlusecase[width=3cm, x = 20, y = -6.75, name=pymulti,fill=orange!20]{UC-04.2: Python multi-core}
            \umlusecase[width=3cm, x = 20, y = -8.5, name=pypysingle,fill=green!20]{UC-05.1: PyPy single-core}
            \umlusecase[width=3cm, x = 20, y = -10.25, name=pypymulti,fill=green!20]{UC-05.2: PyPy multi-core}

            \umlassoc{User}{runall}
            \umlassoc{User}{runcpp}
            \umlassoc{User}{rungo}
            \umlassoc{User}{runpy}
            \umlassoc{User}{runpypy}

            % Include relationships from runall - positioned at top and bottom
            \draw [tikzuml dependency style] (runall) to[bend left=16] node[above] {$\ll \text{extend} \gg$} (single);
            % \umlextend[anchor1=45,anchor2=-225,pos=0.7]{runall}{single}
            \umlextend[anchor1=315,anchor2=135,pos=0.2]{runall}{multi}

            % Include relationships to single-core - staggered positions
            \umlinclude[anchor1=0,anchor2=180,pos=0.6]{runcpp}{single}
            \umlinclude[anchor1=45,anchor2=210,pos=0.4]{rungo}{single}
            \umlinclude[anchor1=60,anchor2=220,pos=0.6]{runpy}{single}
            \umlinclude[anchor1=70,anchor2=230,pos=0.4]{runpypy}{single}

            % Include relationships to multi-core - staggered positions
            \umlinclude[anchor1=-30,anchor2=180,pos=0.4]{runcpp}{multi}
            \umlinclude[anchor1=330,anchor2=190,pos=0.6]{rungo}{multi}
            \umlinclude[anchor1=-60,anchor2=210,pos=0.4]{runpy}{multi}
            \umlinclude[anchor1=270,anchor2=230,pos=0.3]{runpypy}{multi}

            % Inheritance relationships
            \umlinherit[anchor2=180]{single}{cppsingle}
            \umlinherit[anchor2=180]{single}{gosingle}
            \umlinherit[anchor2=180]{single}{pysingle}
            \umlinherit[anchor2=180]{single}{pypysingle}

            \umlinherit[anchor2=180]{multi}{cppmulti}
            \umlinherit[anchor2=180]{multi}{gomulti}
            \umlinherit[anchor2=180]{multi}{pymulti}
            \umlinherit[anchor2=180]{multi}{pypymulti}
        \end{umlsystem}

    \end{tikzpicture}
    \caption{Cleaned-up use-case diagram for the Ray-Tracer Benchmarker}
    \label{fig:use-case-v2}
\end{figure}
\end{landscape}



Use case template:

\begin{table}[H]
    \centering
    \begin{tabular}{l p{10cm}}
        \toprule
        \multicolumn{2}{c}{\textbf{ID:\@ UC-XX}} \\
        \toprule
        \textbf{Name}               &  The name of the use case inside the diagram. \\
        \textbf{Actors}             &  User, System \\
        \textbf{Objective}          &  Brief description of the goal of the use case. \\
        \textbf{Description}        &  Steps the actor has to entail. \\
        \textbf{Preconditions}      &  The user has the system installed and configured. \\
        \textbf{Postconditions}     &  The benchmarks are executed, and the results are stored. \\
    \end{tabular}
    \caption{Use Case UC-XX}\label{tab:uc-xx}
\end{table}

Start of use cases:
\begin{table}[H]
    \centering
    \begin{tabular}{l p{10cm}}
        \toprule
        \multicolumn{2}{c}{\textbf{ID:\@ UC-01}} \\
        \toprule
        \textbf{Name}                         &  Run all benchmarks. \\
        \textbf{Actors}                       &  User \\
        \textbf{Objective}                    &  Running all benchmarks, multicore and single-core versions of all the languages implemented. \\
        \multirow{1}{*}{\textbf{Description}} & \textsl{1.} The user writes the command to run the benchmarks: \texttt{make all}.\\
                                              & \textsl{2.} The user defines the number of cores to be used for the multicore benchmarks, CORES=<num>.\\
                                              & \textsl{3.} The user defines the platform to be used for the benchmarks < (SERVER / macOS / RPI)>=True.\\
                                              & \textsl{4.} Finish the execution of the benchmarks.\\
                                              & \textsl{5.} Open the resulting files.\\
        \textbf{Preconditions}                &  N/A \\
        \textbf{Postconditions}               &  The benchmarks are executed, and the results are stored, and those files are output. \\
    \end{tabular}
    \caption{Use Case UC-01}\label{tab:uc-01}
\end{table}

\begin{table}[H]
    \centering
    \begin{tabular}{l p{10cm}}
        \toprule
        \multicolumn{2}{c}{\textbf{ID:\@ UC-02}} \\
        \toprule
        \textbf{Name}                         &  Run C++ Benchmarks. \\
        \textbf{Actors}                       &  User \\
        \textbf{Objective}                    &  Running all C++ implementation of the benchmarks, multicore and single-core versions. \\
        \multirow{1}{*}{\textbf{Description}} & \textsl{1.} The user writes the command to run the benchmarks: \texttt{make cpp cpp-single}.\\
                                              & \textsl{2.} The user defines the number of cores to be used for the multicore benchmarks, CORES=<num>.\\
                                              & \textsl{3.} The user defines the platform to be used for the benchmarks < (SERVER / macOS / RPI)>=True.\\
                                              & \textsl{4.} Finish the execution of the benchmarks.\\
                                              & \textsl{5.} Open the resulting files.\\
        \textbf{Preconditions}                &  N/A \\
        \textbf{Postconditions}               &  The C++ benchmarks are executed, and the results are stored, and those files are output. \\
    \end{tabular}
    \caption{Use Case UC-02}\label{tab:uc-single-core 02}
\end{table}

\begin{table}[H]
    \centering
    \begin{tabular}{l p{10cm}}
        \toprule
        \multicolumn{2}{c}{\textbf{ID:\@ UC-02.1}} \\
        \toprule
        \textbf{Name}                         &  Run C++ Benchmarks. \\
        \textbf{Actors}                       &  User \\
        \textbf{Objective}                    &  Running the C++ implementation of the single-core benchmarks. \\
        \multirow{1}{*}{\textbf{Description}} & \textsl{1.} The user writes the command to run the benchmarks: \texttt{make cpp-single}.\\
                                              & \textsl{2.} The user defines the platform to be used for the benchmarks < (SERVER / macOS / RPI)>=True.\\
                                              & \textsl{3.} Finish the execution of the benchmarks.\\
                                              & \textsl{4.} Open the resulting files.\\
        \textbf{Preconditions}                &  N/A \\
        \textbf{Postconditions}               &  The C++ single-core benchmark is executed, and the results are stored, and those files are output. \\
    \end{tabular}
    \caption{Use Case UC-02.1}\label{tab:uc-02.1}
\end{table}

\begin{table}[H]
    \centering
    \begin{tabular}{l p{10cm}}
        \toprule
        \multicolumn{2}{c}{\textbf{ID:\@ UC-02.2}} \\
        \toprule
        \textbf{Name}                         &  Run C++ Benchmarks. \\
        \textbf{Actors}                       &  User \\
        \textbf{Objective}                    &  Running the C++ implementation of the multicore benchmarks. \\
        \multirow{1}{*}{\textbf{Description}} & \textsl{1.} The user writes the command to run the benchmarks: \texttt{make cpp}.\\
                                              & \textsl{2.} The user defines the number of cores to be used for the multicore benchmarks, CORES=<num>.\\
                                              & \textsl{3.} The user defines the platform to be used for the benchmarks < (SERVER / macOS / RPI)>=True.\\
                                              & \textsl{4.} Finish the execution of the benchmarks.\\
                                              & \textsl{5.} Open the resulting files.\\
        \textbf{Preconditions}                &  N/A \\
        \textbf{Postconditions}               &  The C++ multicore benchmark is executed, and the results are stored, and those files are output. \\
    \end{tabular}
    \caption{Use Case UC-02.2}\label{tab:uc-02.2}
\end{table}



\begin{table}[H]
    \centering
    \begin{tabular}{l p{10cm}}
        \toprule
        \multicolumn{2}{c}{\textbf{ID:\@ UC-03}} \\
        \toprule
        \textbf{Name}                         &  Run Go Benchmarks. \\
        \textbf{Actors}                       &  User \\
        \textbf{Objective}                    &  Running all Go implementation of the benchmarks, multicore and single-core versions. \\
        \multirow{1}{*}{\textbf{Description}} & \textsl{1.} The user writes the command to run the benchmarks: \texttt{make go go-single}.\\
                                              & \textsl{2.} The user defines the number of cores to be used for the multicore benchmarks, CORES=<num>.\\
                                              & \textsl{3.} The user defines the platform to be used for the benchmarks < (SERVER / macOS / RPI)>=True.\\
                                              & \textsl{4.} Finish the execution of the benchmarks.\\
                                              & \textsl{5.} Open the resulting files.\\
        \textbf{Preconditions}                &  N/A \\
        \textbf{Postconditions}               &  The Go benchmarks are executed, and the results are stored, and those files are output. \\
    \end{tabular}
    \caption{Use Case UC-03}\label{tab:uc-03}
\end{table}


\begin{table}[H]
    \centering
    \begin{tabular}{l p{10cm}}
        \toprule
        \multicolumn{2}{c}{\textbf{ID:\@ UC-03.1}} \\
        \toprule
        \textbf{Name}                         &  Run Go Benchmarks. \\
        \textbf{Actors}                       &  User \\
        \textbf{Objective}                    &  Running the Go implementation of the single-core benchmarks. \\
        \multirow{1}{*}{\textbf{Description}} & \textsl{1.} The user writes the command to run the benchmarks: \texttt{make go-single}.\\
                                              & \textsl{2.} The user defines the platform to be used for the benchmarks < (SERVER / macOS / RPI)>=True.\\
                                              & \textsl{3.} Finish the execution of the benchmarks.\\
                                              & \textsl{4.} Open the resulting files.\\
        \textbf{Preconditions}                &  N/A \\
        \textbf{Postconditions}               &  The Go single-core benchmark is executed, and the results are stored, and those files are output. \\
    \end{tabular}
    \caption{Use Case UC-03.1}\label{tab:uc-03.1}
\end{table}

\begin{table}[H]
    \centering
    \begin{tabular}{l p{10cm}}
        \toprule
        \multicolumn{2}{c}{\textbf{ID:\@ UC-03.2}} \\
        \toprule
        \textbf{Name}                         &  Run Go Benchmarks. \\
        \textbf{Actors}                       &  User \\
        \textbf{Objective}                    &  Running the Go implementation of the multicore benchmarks. \\
        \multirow{1}{*}{\textbf{Description}} & \textsl{1.} The user writes the command to run the benchmarks: \texttt{make go}.\\
                                              & \textsl{2.} The user defines the number of cores to be used for the multicore benchmarks, CORES=<num>.\\
                                              & \textsl{3.} The user defines the platform to be used for the benchmarks < (SERVER / macOS / RPI)>=True.\\
                                              & \textsl{4.} Finish the execution of the benchmarks.\\
                                              & \textsl{5.} Open the resulting files.\\
        \textbf{Preconditions}                &  N/A \\
        \textbf{Postconditions}               &  The Go multicore benchmark is executed, and the results are stored, and those files are output. \\
    \end{tabular}
    \caption{Use Case UC-03.2}\label{tab:uc-03.2}
\end{table}


\begin{table}[H]
    \centering
    \begin{tabular}{l p{10cm}}
        \toprule
        \multicolumn{2}{c}{\textbf{ID:\@ UC-04}} \\
        \toprule
        \textbf{Name}                         &  Run Python Benchmarks. \\
        \textbf{Actors}                       &  User \\
        \textbf{Objective}                    &  Running all Python implementation of the benchmarks, multicore and single-core versions. \\
        \multirow{1}{*}{\textbf{Description}} & \textsl{1.} The user writes the command to run the benchmarks: \texttt{make python python-single}.\\
                                              & \textsl{2.} The user defines the number of cores to be used for the multicore benchmarks, CORES=<num>.\\
                                              & \textsl{3.} The user defines the platform to be used for the benchmarks < (SERVER / macOS / RPI)>=True.\\
                                              & \textsl{4.} Finish the execution of the benchmarks.\\
                                              & \textsl{5.} Open the resulting files.\\
        \textbf{Preconditions}                &  N/A \\
        \textbf{Postconditions}               &  The Python benchmarks are executed, and the results are stored, and those files are output. \\
    \end{tabular}
    \caption{Use Case UC-04}\label{tab:uc-04}
\end{table}


\begin{table}[H]
    \centering
    \begin{tabular}{l p{10cm}}
        \toprule
        \multicolumn{2}{c}{\textbf{ID:\@ UC-04.1}} \\
        \toprule
        \textbf{Name}                         &  Run Python Benchmarks. \\
        \textbf{Actors}                       &  User \\
        \textbf{Objective}                    &  Running the Python implementation of the single-core benchmarks. \\
        \multirow{1}{*}{\textbf{Description}} & \textsl{1.} The user writes the command to run the benchmarks: \texttt{make python-single}.\\
                                              & \textsl{2.} The user defines the platform to be used for the benchmarks < (SERVER / macOS / RPI)>=True.\\
                                              & \textsl{3.} Finish the execution of the benchmarks.\\
                                              & \textsl{4.} Open the resulting files.\\
        \textbf{Preconditions}                &  N/A \\
        \textbf{Postconditions}               &  The Python single-core benchmark is executed, and the results are stored, and those files are output. \\
    \end{tabular}
    \caption{Use Case UC-04.1}\label{tab:uc-04.1}
\end{table}

\begin{table}[H]
    \centering
    \begin{tabular}{l p{10cm}}
        \toprule
        \multicolumn{2}{c}{\textbf{ID:\@ UC-04.2}} \\
        \toprule
        \textbf{Name}                         &  Run Python Benchmarks. \\
        \textbf{Actors}                       &  User \\
        \textbf{Objective}                    &  Running the Python implementation of the multicore benchmarks. \\
        \multirow{1}{*}{\textbf{Description}} & \textsl{1.} The user writes the command to run the benchmarks: \texttt{make python}.\\
                                              & \textsl{2.} The user defines the number of cores to be used for the multicore benchmarks, CORES=<num>.\\
                                              & \textsl{3.} The user defines the platform to be used for the benchmarks < (SERVER / macOS / RPI)>=True.\\
                                              & \textsl{4.} Finish the execution of the benchmarks.\\
                                              & \textsl{5.} Open the resulting files.\\
        \textbf{Preconditions}                &  N/A \\
        \textbf{Postconditions}               &  The Python multicore benchmark is executed, and the results are stored, and those files are output. \\
    \end{tabular}
    \caption{Use Case UC-04.2}\label{tab:uc-04.2}
\end{table}


\begin{table}[H]
    \centering
    \begin{tabular}{l p{10cm}}
        \toprule
        \multicolumn{2}{c}{\textbf{ID:\@ UC-05}} \\
        \toprule
        \textbf{Name}                         &  Run PyPy Benchmarks. \\
        \textbf{Actors}                       &  User \\
        \textbf{Objective}                    &  Running all PyPy implementation of the benchmarks, multicore and single-core versions. \\
        \multirow{1}{*}{\textbf{Description}} & \textsl{1.} The user writes the command to run the benchmarks: \texttt{make pypy pypy-single}.\\
                                              & \textsl{2.} The user defines the number of cores to be used for the multicore benchmarks, CORES=<num>.\\
                                              & \textsl{3.} The user defines the platform to be used for the benchmarks < (SERVER / macOS / RPI)>=True.\\
                                              & \textsl{4.} Finish the execution of the benchmarks.\\
                                              & \textsl{5.} Open the resulting files.\\
        \textbf{Preconditions}                &  N/A \\
        \textbf{Postconditions}               &  The PyPy benchmarks are executed, and the results are stored, and those files are output. \\
    \end{tabular}
    \caption{Use Case UC-05}\label{tab:uc-05}
\end{table}


\begin{table}[H]
    \centering
    \begin{tabular}{l p{10cm}}
        \toprule
        \multicolumn{2}{c}{\textbf{ID:\@ UC-05.1}} \\
        \toprule
        \textbf{Name}                         &  Run PyPy Benchmarks. \\
        \textbf{Actors}                       &  User \\
        \textbf{Objective}                    &  Running the PyPy implementation of the single-core benchmarks. \\
        \multirow{1}{*}{\textbf{Description}} & \textsl{1.} The user writes the command to run the benchmarks: \texttt{make pypy-single}.\\
                                              & \textsl{2.} The user defines the platform to be used for the benchmarks < (SERVER / macOS / RPI)>=True.\\
                                              & \textsl{3.} Finish the execution of the benchmarks.\\
                                              & \textsl{4.} Open the resulting files.\\
        \textbf{Preconditions}                &  N/A \\
        \textbf{Postconditions}               &  The PyPy single-core benchmark is executed, and the results are stored, and those files are output. \\
    \end{tabular}
    \caption{Use Case UC-05.1}\label{tab:uc-05.1}
\end{table}

\begin{table}[H]
    \centering
    \begin{tabular}{l p{10cm}}
        \toprule
        \multicolumn{2}{c}{\textbf{ID:\@ UC-05.2}} \\
        \toprule
        \textbf{Name}                         &  Run PyPy Benchmarks. \\
        \textbf{Actors}                       &  User \\
        \textbf{Objective}                    &  Running the PyPy implementation of the multicore benchmarks. \\
        \multirow{1}{*}{\textbf{Description}} & \textsl{1.} The user writes the command to run the benchmarks: \texttt{make pypy}.\\
                                              & \textsl{2.} The user defines the number of cores to be used for the multicore benchmarks, CORES=<num>.\\
                                              & \textsl{3.} The user defines the platform to be used for the benchmarks < (SERVER / macOS / RPI)>=True.\\
                                              & \textsl{4.} Finish the execution of the benchmarks.\\
                                              & \textsl{5.} Open the resulting files.\\
        \textbf{Preconditions}                &  N/A \\
        \textbf{Postconditions}               &  The PyPy multicore benchmark is executed, and the results are stored, and those files are output. \\
    \end{tabular}
    \caption{Use Case UC-05.2}\label{tab:uc-05.2}
\end{table}



\begin{table}[H]
    \centering
    \begin{tabular}{l p{10cm}}
        \toprule
        \multicolumn{2}{c}{\textbf{ID:\@ UC-06}} \\
        \toprule
        \textbf{Name}                         &  Run Single core Benchmarks. \\
        \textbf{Actors}                       &  User \\
        \textbf{Objective}                    &  Running all PyPy implementation of the single-core version of the benchmarks. \\
        \multirow{1}{*}{\textbf{Description}} & \textsl{1.} The user writes the command to run the benchmarks: \texttt{make all-single}.\\
                                              & \textsl{2.} The user defines the platform to be used for the benchmarks < (SERVER / macOS / RPI)>=True.\\
                                              & \textsl{3.} Finish the execution of the benchmarks.\\
                                              & \textsl{4.} Open the resulting files.\\
        \textbf{Preconditions}                &  N/A \\
        \textbf{Postconditions}               &  The single-core implementation of the benchmarks are executed, and the results are stored, and those files are output. \\
    \end{tabular}
    \caption{Use Case UC-06}\label{tab:uc-06}
\end{table}

\begin{table}[H]
    \centering
    \begin{tabular}{l p{10cm}}
        \toprule
        \multicolumn{2}{c}{\textbf{ID:\@ UC-07}} \\
        \toprule
        \textbf{Name}                         &  Run multicore Benchmarks. \\
        \textbf{Actors}                       &  User \\
        \textbf{Objective}                    &  Running all PyPy implementation of the multicore version of the benchmarks. \\
        \multirow{1}{*}{\textbf{Description}} & \textsl{1.} The user writes the command to run the benchmarks: \texttt{make multi}.\\
                                              & \textsl{2.} The user defines the number of cores to be used for the multicore benchmarks, CORES=<num>.\\
                                              & \textsl{3.} The user defines the platform to be used for the benchmarks < (SERVER / macOS / RPI)>=True.\\
                                              & \textsl{4.} Finish the execution of the benchmarks.\\
                                              & \textsl{5.} Open the resulting files.\\
        \textbf{Preconditions}                &  N/A \\
        \textbf{Postconditions}               &  The multicore implementation of the benchmarks are executed, and the results are stored, and those files are output. \\
    \end{tabular}
    \caption{Use Case UC-07}\label{tab:uc-07}
\end{table}



\section{Traceability}

This section provides a traceability matrix that maps the requirements to the use cases. The matrix is designed to ensure that all requirements are addressed by the use cases defined in this document.

It can be seen in \autoref{tab:traceability-matrix-ur} that all requirements are covered by the use cases, ensuring that the system meets the specified needs and in the following table \autoref{tab:traceability-matrix-fn} the functional and non-functional requirements are mapped to the use cases.

\begin{landscape}

\begin{table}[t]
    \centering
    \begin{tabular}{|c|c|c|c|c|c|c|c|c|c|c|c|c|c|c|c|}
        \hline
        \textbf{Requirement} & \rotatebox{65}{\textbf{UC-01}} & \rotatebox{65}{\textbf{UC-02}} & \rotatebox{65}{\textbf{UC-02.}1} & \rotatebox{65}{\textbf{UC-02.}2} & \rotatebox{65}{\textbf{UC-03}} & \rotatebox{65}{\textbf{UC-03.}1} & \rotatebox{65}{\textbf{UC-03.}2} & \rotatebox{65}{\textbf{UC-04}} & \rotatebox{65}{\textbf{UC-04.}1} & \rotatebox{65}{\textbf{UC-04.}2} & \rotatebox{65}{\textbf{UC-05}} & \rotatebox{65}{\textbf{UC-05.}1} & \rotatebox{65}{\textbf{UC-05.}2} & \rotatebox{65}{\textbf{UC-06}} & \rotatebox{65}{\textbf{UC-07}} \\
        \hline
        UR-CA-01 & \checkmark & \checkmark & \checkmark & \checkmark & \checkmark & \checkmark & \checkmark & \checkmark & \checkmark & \checkmark & \checkmark & \checkmark & \checkmark & \checkmark & \checkmark \\
        \hline
        UR-CA-02 &            &            &            &            &            &            &            &            &            &            &            &            &            &            &            \\
        \hline
        UR-CA-03 & \checkmark &            &            &            &            &            &            &            &            &            &            &            &            &            &            \\
        \hline
        UR-CA-04 & \checkmark & \checkmark & \checkmark & \checkmark & \checkmark & \checkmark & \checkmark & \checkmark & \checkmark & \checkmark & \checkmark & \checkmark & \checkmark & \checkmark & \checkmark \\
        \hline
        UR-CA-05 & \checkmark & \checkmark & \checkmark & \checkmark & \checkmark & \checkmark & \checkmark & \checkmark & \checkmark & \checkmark & \checkmark & \checkmark & \checkmark & \checkmark & \checkmark \\
        \hline
        %% Multic-core
        UR-CA-06 & \checkmark & \checkmark &            & \checkmark & \checkmark &            & \checkmark & \checkmark &            & \checkmark & \checkmark &            & \checkmark &            & \checkmark \\
        \hline
        UR-CA-07 & \checkmark & \checkmark & \checkmark & \checkmark & \checkmark & \checkmark & \checkmark & \checkmark & \checkmark & \checkmark & \checkmark & \checkmark & \checkmark & \checkmark & \checkmark \\
        \hline
        UR-CA-08 & \checkmark & \checkmark & \checkmark & \checkmark & \checkmark & \checkmark & \checkmark & \checkmark & \checkmark & \checkmark & \checkmark & \checkmark & \checkmark & \checkmark & \checkmark \\
        \hline
        UR-CA-09 & \checkmark & \checkmark & \checkmark & \checkmark & \checkmark & \checkmark & \checkmark & \checkmark & \checkmark & \checkmark & \checkmark & \checkmark & \checkmark & \checkmark & \checkmark \\
        \hline
        UR-CA-10 & \checkmark & \checkmark & \checkmark & \checkmark & \checkmark & \checkmark & \checkmark & \checkmark & \checkmark & \checkmark & \checkmark & \checkmark & \checkmark & \checkmark & \checkmark \\
        \hline
        UR-CA-11 & \checkmark & \checkmark & \checkmark & \checkmark & \checkmark & \checkmark & \checkmark & \checkmark & \checkmark & \checkmark & \checkmark & \checkmark & \checkmark & \checkmark & \checkmark \\
        \hline
        UR-CA-12 &            &            &            &            &            &            &            &            &            &            &            &            &            &            &            \\
        \hline
        UR-CA-13 & \checkmark & \checkmark &            & \checkmark & \checkmark &            & \checkmark & \checkmark &            & \checkmark & \checkmark &            & \checkmark &            & \checkmark \\
        \hline
        UR-CA-14 & \checkmark & \checkmark & \checkmark & \checkmark & \checkmark & \checkmark & \checkmark & \checkmark & \checkmark & \checkmark & \checkmark & \checkmark & \checkmark & \checkmark & \checkmark \\
        \hline
        UR-CA-15 & \checkmark & \checkmark & \checkmark & \checkmark & \checkmark & \checkmark & \checkmark & \checkmark & \checkmark & \checkmark & \checkmark & \checkmark & \checkmark & \checkmark & \checkmark \\
        \hline
        UR-CA-16 &            &            & \checkmark & \checkmark &            & \checkmark & \checkmark &            & \checkmark & \checkmark &            & \checkmark & \checkmark &            &            \\
        \hline
        UR-CA-17 &            & \checkmark &            &            & \checkmark &            &            & \checkmark &            &            & \checkmark &            &            &            &            \\
        \hline
        UR-RE-01 & \checkmark & \checkmark & \checkmark & \checkmark & \checkmark & \checkmark & \checkmark & \checkmark & \checkmark & \checkmark & \checkmark & \checkmark & \checkmark & \checkmark & \checkmark \\
        \hline
        UR-RE-02 &            & \checkmark &            &            & \checkmark &            &            & \checkmark &            &            & \checkmark &            &            &            &            \\
        \hline

    \end{tabular}
\caption{Traceability matrix (for UC-CA \& UC-RE requirement)}\label{tab:traceability-matrix-ur}
\end{table}


\begin{table}
    \centering
    \begin{tabular}{|c|c|c|c|c|c|c|c|c|c|c|c|c|c|c|c|c}
        \hline
        \textbf{Requirement} & \rotatebox{65}{\textbf{UC-01}} & \rotatebox{65}{\textbf{UC-02}} & \rotatebox{65}{\textbf{UC-02.}1} & \rotatebox{65}{\textbf{UC-02.}2} & \rotatebox{65}{\textbf{UC-03}} & \rotatebox{65}{\textbf{UC-03.}1} & \rotatebox{65}{\textbf{UC-03.}2} & \rotatebox{65}{\textbf{UC-04}} & \rotatebox{65}{\textbf{UC-04.}1} & \rotatebox{65}{\textbf{UC-04.}2} & \rotatebox{65}{\textbf{UC-05}} & \rotatebox{65}{\textbf{UC-05.}1} & \rotatebox{65}{\textbf{UC-05.}2} & \rotatebox{65}{\textbf{UC-06}} & \rotatebox{65}{\textbf{UC-07}} \\
        \hline
        SR-NF-01 & \checkmark & \checkmark & \checkmark & \checkmark & \checkmark & \checkmark & \checkmark & \checkmark & \checkmark & \checkmark & \checkmark & \checkmark & \checkmark & \checkmark & \checkmark \\
        \hline
        SR-NF-02 & \checkmark &            &            &            &            &            &            &            &            &            &            &            &            &            &            \\
        \hline
        SR-NF-03 & \checkmark & \checkmark & \checkmark & \checkmark &            &            &            &            &            &            &            &            &            &            &            \\
        \hline
        SR-NF-04 & \checkmark &            &            &            & \checkmark & \checkmark & \checkmark &            &            &            &            &            &            &            &            \\
        \hline
        SR-NF-05 & \checkmark &            &            &            &            &            &            & \checkmark & \checkmark & \checkmark &            &            &            &            &            \\
        \hline
        SR-NF-06 & \checkmark & \checkmark & \checkmark & \checkmark & \checkmark & \checkmark & \checkmark & \checkmark & \checkmark & \checkmark & \checkmark & \checkmark & \checkmark & \checkmark & \checkmark \\
        \hline
        SR-NF-07 & \checkmark & \checkmark & \checkmark & \checkmark & \checkmark & \checkmark & \checkmark & \checkmark & \checkmark & \checkmark & \checkmark & \checkmark & \checkmark & \checkmark & \checkmark \\
        \hline
        SR-NF-08 & \checkmark & \checkmark & \checkmark & \checkmark & \checkmark & \checkmark & \checkmark & \checkmark & \checkmark & \checkmark & \checkmark & \checkmark & \checkmark & \checkmark & \checkmark \\
        \hline
        %% functional requirements
        SR-FN-01 & \checkmark & \checkmark & \checkmark & \checkmark & \checkmark & \checkmark & \checkmark & \checkmark & \checkmark & \checkmark & \checkmark & \checkmark & \checkmark & \checkmark & \checkmark \\
        \hline
        SR-FN-02 & \checkmark & \checkmark & \checkmark & \checkmark & \checkmark & \checkmark & \checkmark & \checkmark & \checkmark & \checkmark & \checkmark & \checkmark & \checkmark & \checkmark & \checkmark \\
        \hline
        SR-FN-03 & \checkmark & \checkmark & \checkmark & \checkmark & \checkmark & \checkmark & \checkmark & \checkmark & \checkmark & \checkmark & \checkmark & \checkmark & \checkmark & \checkmark & \checkmark \\
        \hline
        SR-FN-04 & \checkmark & \checkmark & \checkmark & \checkmark & \checkmark & \checkmark & \checkmark & \checkmark & \checkmark & \checkmark & \checkmark & \checkmark & \checkmark & \checkmark & \checkmark \\
        \hline
        SR-FN-05 & \checkmark & \checkmark & \checkmark & \checkmark & \checkmark & \checkmark & \checkmark & \checkmark & \checkmark & \checkmark & \checkmark & \checkmark & \checkmark & \checkmark & \checkmark \\
        \hline
        SR-FN-06 & \checkmark & \checkmark & \checkmark & \checkmark & \checkmark & \checkmark & \checkmark & \checkmark & \checkmark & \checkmark & \checkmark & \checkmark & \checkmark & \checkmark & \checkmark \\
        \hline
        SR-FN-07 & \checkmark & \checkmark & \checkmark & \checkmark & \checkmark & \checkmark & \checkmark & \checkmark & \checkmark & \checkmark & \checkmark & \checkmark & \checkmark & \checkmark & \checkmark \\
        \hline
        SR-FN-08 & \checkmark & \checkmark & \checkmark & \checkmark & \checkmark & \checkmark & \checkmark & \checkmark & \checkmark & \checkmark & \checkmark & \checkmark & \checkmark & \checkmark & \checkmark \\
        \hline
        SR-FN-09 &            &            &            &            &            &            &            &            &            &            &            &            &            &            &            \\
        \hline
        SR-FN-10 & \checkmark & \checkmark & \checkmark & \checkmark & \checkmark & \checkmark & \checkmark & \checkmark & \checkmark & \checkmark & \checkmark & \checkmark & \checkmark & \checkmark & \checkmark \\
        \hline
        SR-FN-11 & \checkmark & \checkmark & \checkmark & \checkmark & \checkmark & \checkmark & \checkmark & \checkmark & \checkmark & \checkmark & \checkmark & \checkmark & \checkmark & \checkmark & \checkmark \\
        \hline

    \end{tabular}
    \caption{Traceability matrix (for functional \& non-functional requirements)}\label{tab:traceability-matrix-fn}
\end{table}

\end{landscape}
