\chapter{Problem Statement}\label{chap:analysis}

\section{Project Description}
Raytracer in multiple languages, comparing the energy efficiency of each one.

\section{Requirements}

UR -> User Requirement
CA -> Capacity
RE -> Restriction

\subsection{User Requirements}

\subsubsection{Capacity}

\begin{table}[H]
    \centering
    \begin{tabular}{l p{10cm}}
        \toprule
        \multicolumn{2}{c}{UR-CA-XX} \\
        \toprule
        \textbf{Description}        & \textit{Requirement's Description} \\
        \textbf{Necessity}          & \textit{Low / Medium / High} \\
        \textbf{Priority}           & \textit{Low / Medium / High} \\
        \multirow{1}{*}{\textbf{Stability}} & \textit{Stable / Unstable}: How easy it is for the requirement to change \\ 
                                            & along the development of the project \\
        \textbf{Verifiability}       & \textit{Verifiable / Non-Verifiable} \\
    \end{tabular}
    \caption{Requirement UR-CA-XX}
    \label{tab:ur-ca-xx}
\end{table}

\begin{table}[H]
    \centering
    \begin{tabular}{l p{10cm}}
        \toprule
        \multicolumn{2}{c}{UR-CA-01} \\
        \toprule
        \textbf{Description}        &  The user must be able to run the language benchmarks on MacOS and Linux.  \\
        \textbf{Necessity}          &  High   \\
        \textbf{Priority}           &  Medium   \\
        \textbf{Stability}          &  Stable   \\
        \textbf{Verifiability}       &  Verifiable \\
    \end{tabular}
    \caption{Requirement UR-CA-01}
    \label{tab:ur-ca-01}
\end{table}

\begin{table}[H]
    \centering
    \begin{tabular}{l p{10cm}}
        \toprule
        \multicolumn{2}{c}{UR-CA-02} \\
        \toprule
        \textbf{Description}        & The user must be able to inspect every aspect of the code. \\
        \textbf{Necessity}          & High  \\
        \textbf{Priority}           & High  \\
        \textbf{Stability}          & Stable \\
        \textbf{Verifiability}       & Verifiable \\
    \end{tabular}
    \caption{Requirement UR-CA-02}
    \label{tab:ur-ca-02}
\end{table}

\begin{table}[H]
    \centering
    \begin{tabular}{l p{10cm}}
        \toprule
        \multicolumn{2}{c}{UR-CA-03} \\
        \toprule
        \textbf{Description}        & The user must be able to add their own implementation of the program to be tested. \\
        \textbf{Necessity}          &  High   \\
        \textbf{Priority}           &  High   \\
        \textbf{Stability}          &  Stable \\
        \textbf{Verifiability}       &  Verifiable \\
    \end{tabular}
    \caption{Requirement UR-CA-03}
    \label{tab:ur-ca-03}
\end{table}

\begin{table}[H]
    \centering
    \begin{tabular}{l p{10cm}}
        \toprule
        \multicolumn{2}{c}{UR-CA-04} \\
        \toprule
        \textbf{Description}        & The user must be able to check the outputs of the programs. \\
        \textbf{Necessity}          & High   \\
        \textbf{Priority}           & High   \\
        \textbf{Stability}          & Stable \\
        \textbf{Verifiability}      & Verifiable \\
    \end{tabular}
    \caption{Requirement UR-CA-04}
    \label{tab:ur-ca-04}
\end{table}

\begin{table}[H]
    \centering
    \begin{tabular}{l p{10cm}}
        \toprule
        \multicolumn{2}{c}{UR-CA-05} \\
        \toprule
        \textbf{Description}        & The user must be able to check the enery consumption of the benchmarks per-core\\
        \textbf{Necessity}          & High   \\
        \textbf{Priority}           & High   \\
        \textbf{Stability}          & Stable \\
        \textbf{Verifiability}      & Verifiable \\
    \end{tabular}
    \caption{Requirement UR-CA-05}
    \label{tab:ur-ca-05}
\end{table}

\begin{table}[H]
    \centering
    \begin{tabular}{l p{10cm}}
        \toprule
        \multicolumn{2}{c}{UR-CA-06} \\
        \toprule
        \textbf{Description}        & The user must be able to decide the amount of cores needed via the CLI\\
        \textbf{Necessity}          & High   \\
        \textbf{Priority}           & High   \\
        \textbf{Stability}          & Stable \\
        \textbf{Verifiability}      & Verifiable \\
    \end{tabular}
    \caption{Requirement UR-CA-06}
    \label{tab:ur-ca-06}
\end{table}

\begin{table}[H]
    \centering
    \begin{tabular}{l p{10cm}}
        \toprule
        \multicolumn{2}{c}{UR-CA-07} \\
        \toprule
        \textbf{Description}        & The user must be able to specify if the banchmark should use \texttt{taskset} to fix the number of avaiable cores.\\
        \textbf{Necessity}          & High   \\
        \textbf{Priority}           & High   \\
        \textbf{Stability}          & Stable \\
        \textbf{Verifiability}      & Verifiable \\
    \end{tabular}
    \caption{Requirement UR-CA-07}
    \label{tab:ur-ca-07}
\end{table}

\begin{table}[H]
    \centering
    \begin{tabular}{l p{10cm}}
        \toprule
        \multicolumn{2}{c}{UR-CA-08} \\
        \toprule
        \textbf{Description}        & The user should be able to modify the execution command to add specific performance metric and energy consumption parameters before the execution of the benchmark. \\
        \textbf{Necessity}          & Low   \\
        \textbf{Priority}           & Low   \\
        \textbf{Stability}          & Unstable \\
        \textbf{Verifiability}      & Verifiable \\
    \end{tabular}
    \caption{Requirement UR-CA-08}
    \label{tab:ur-ca-08}
\end{table}

\begin{table}[H]
    \centering
    \begin{tabular}{l p{10cm}}
        \toprule
        \multicolumn{2}{c}{UR-CA-09} \\
        \toprule
        \textbf{Description}        & The user must be able to check the time taken to run the different programs / benchmarks \\
        \textbf{Necessity}          & High   \\
        \textbf{Priority}           & High   \\
        \textbf{Stability}          & Stable \\
        \textbf{Verifiability}      & Verifiable \\
    \end{tabular}
    \caption{Requirement UR-CA-09}
    \label{tab:ur-ca-09}
\end{table}

\begin{table}[H]
    \centering
    \begin{tabular}{l p{10cm}}
        \toprule
        \multicolumn{2}{c}{UR-CA-10} \\
        \toprule
        \textbf{Description}        & The user must be able to change the parameters of the image generated \\
        \textbf{Necessity}          & High   \\
        \textbf{Priority}           & High   \\
        \textbf{Stability}          & Stable \\
        \textbf{Verifiability}      & Verifiable \\
    \end{tabular}
    \caption{Requirement UR-CA-10}
    \label{tab:ur-ca-10}
\end{table}

\begin{table}[H]
    \centering
    \begin{tabular}{l p{10cm}}
        \toprule
        \multicolumn{2}{c}{UR-CA-11} \\
        \toprule
        \textbf{Description}        & The user must be able to visualize the resulting images of the program execution \\
        \textbf{Necessity}          & High   \\
        \textbf{Priority}           & High   \\
        \textbf{Stability}          & Stable \\
        \textbf{Verifiability}      & Verifiable \\
    \end{tabular}
    \caption{Requirement UR-CA-11}
    \label{tab:ur-ca-11}
\end{table}

\begin{table}[H]
    \centering
    \begin{tabular}{l p{10cm}}
        \toprule
        \multicolumn{2}{c}{UR-CA-12} \\
        \toprule
        \textbf{Description}        & The user must be able to compare the resulting images of the program execution \\
        \textbf{Necessity}          & High   \\
        \textbf{Priority}           & High   \\
        \textbf{Stability}          & Stable \\
        \textbf{Verifiability}      & Verifiable \\
    \end{tabular}
    \caption{Requirement UR-CA-12}
    \label{tab:ur-ca-12}
\end{table}

\begin{table}[H]
    \centering
    \begin{tabular}{l p{10cm}}
        \toprule
        \multicolumn{2}{c}{UR-CA-13} \\
        \toprule
        \textbf{Description}        & The system must inform the user how many cores it is using though the terminal \\
        \textbf{Necessity}          & Medium   \\
        \textbf{Priority}           & Low   \\
        \textbf{Stability}          & Stable \\
        \textbf{Verifiability}      & Verifiable \\
    \end{tabular}
    \caption{Requirement UR-CA-13}
    \label{tab:ur-ca-13}
\end{table}

\begin{table}[H]
    \centering
    \begin{tabular}{l p{10cm}}
        \toprule
        \multicolumn{2}{c}{UR-CA-14} \\
        \toprule
        \textbf{Description}        & The user must be able to compare different executions on the same platform. \\
        \textbf{Necessity}          & High   \\
        \textbf{Priority}           & High   \\
        \textbf{Stability}          & Stable \\
        \textbf{Verifiability}      & Verifiable \\
    \end{tabular}
    \caption{Requirement UR-CA-14}
    \label{tab:ur-ca-14}
\end{table}

\begin{table}[H]
    \centering
    \begin{tabular}{l p{10cm}}
        \toprule
        \multicolumn{2}{c}{UR-CA-15} \\
        \toprule
        \textbf{Description}        & The user must be able to view the remaining lines of the image to be renderes. \\
        \textbf{Necessity}          & Low   \\
        \textbf{Priority}           & Low   \\
        \textbf{Stability}          & Stable \\
        \textbf{Verifiability}      & Verifiable \\
    \end{tabular}
    \caption{Requirement UR-CA-15}
    \label{tab:ur-ca-15}
\end{table}

\begin{table}[H]
    \centering
    \begin{tabular}{l p{10cm}}
        \toprule
        \multicolumn{2}{c}{UR-CA-16} \\
        \toprule
        \textbf{Description}        & The user must be able to run each benchmark individually. \\
        \textbf{Necessity}          & Medium   \\
        \textbf{Priority}           & Low   \\
        \textbf{Stability}          & Stable \\
        \textbf{Verifiability}      & Verifiable \\
    \end{tabular}
    \caption{Requirement UR-CA-16}
    \label{tab:ur-ca-16}
\end{table}

\begin{table}[H]
    \centering
    \begin{tabular}{l p{10cm}}
        \toprule
        \multicolumn{2}{c}{UR-CA-17} \\
        \toprule
        \textbf{Description}        & The user must be able to run all benchmarks of one language together. \\
        \textbf{Necessity}          & Medium   \\
        \textbf{Priority}           & Low   \\
        \textbf{Stability}          & Stable \\
        \textbf{Verifiability}      & Verifiable \\
    \end{tabular}
    \caption{Requirement UR-CA-17}
    \label{tab:ur-ca-17}
\end{table}

\subsubsection{Restriction}

\begin{table}[H]
    \centering
    \begin{tabular}{l p{10cm}}
        \toprule
        \multicolumn{2}{c}{UR-RE-01} \\
        \toprule
        \textbf{Description}        &  The same architecture should be used for every program \\
        \textbf{Necessity}          &  Low \\
        \textbf{Priority}           &  Low \\
        \textbf{Stability}          &  Unstable \\
        \textbf{Verifiability}      &  Verifiable \\
    \end{tabular}
    \caption{Requirement UR-RE-01}
    \label{tab:ur-re-01}
\end{table}

\begin{table}[H]
    \centering
    \begin{tabular}{l p{10cm}}
        \toprule
        \multicolumn{2}{c}{UR-RE-02} \\
        \toprule
        \textbf{Description}        & The system must have a "one-command" execution for any given language \\
        \textbf{Necessity}          & Medium   \\
        \textbf{Priority}           & Low   \\
        \textbf{Stability}          & Stable \\
        \textbf{Verifiability}      & Verifiable \\
    \end{tabular}
    \caption{Requirement UR-RE-02}
    \label{tab:ur-re-02}
\end{table}


\section{Functional \& Non Functional Requirements}

\subsection{Non Functional Requirements}
\begin{table}[H]
    \centering
    \begin{tabular}{l p{10cm}}
        \toprule
        \multicolumn{2}{c}{SR-NF-01} \\
        \toprule
        \textbf{Description}        & The different programs must have the same architecture \\
        \textbf{Necessity}          &  High \\
        \textbf{Priority}           &  High \\
        \textbf{Stability}          &  Stable \\
        \textbf{Verifiability}      & Verifiable \\
        \textbf{Origin}             &  \textit{\nameref{tab:ur-re-01}}, \textit{\nameref{tab:ur-ca-14}} \\
    \end{tabular}
    \caption{Requirement SR-NF-01}
    \label{tab:sr-nf-01}
\end{table}

\begin{table}[H]
    \centering
    \begin{tabular}{l p{10cm}}
        \toprule
        \multicolumn{2}{c}{SR-NF-02} \\
        \toprule
        \textbf{Description}        & The system should be open-source \\
        \textbf{Necessity}          &  High \\
        \textbf{Priority}           &  High \\
        \textbf{Stability}          &  Stable \\
        \textbf{Verifiability}      & Verifiable \\
        \textbf{Origin}             &  \textit{\nameref{tab:ur-re-02}}, \textit{\nameref{tab:ur-ca-03}}, \textit{\nameref{tab:ur-ca-05}}, \textit{\nameref{tab:ur-ca-06}}, \textit{\nameref{tab:ur-ca-07}}, \textit{\nameref{tab:ur-ca-08}}, \textit{\nameref{tab:ur-ca-10}}, \textit{\nameref{tab:ur-ca-12}}, \textit{\nameref{tab:ur-ca-13}} \\
    \end{tabular}
    \caption{Requirement SR-NF-02}
    \label{tab:sr-nf-02}
\end{table}

\begin{table}[H]
    \centering
    \begin{tabular}{l p{10cm}}
        \toprule
        \multicolumn{2}{c}{SR-NF-03} \\
        \toprule
        \textbf{Description}        & The ray-tracer should be implemented in C++ \\
        \textbf{Necessity}          &  High \\
        \textbf{Priority}           &  High \\
        \textbf{Stability}          &  Stable \\
        \textbf{Verifiability}      & Verifiable \\
        \textbf{Origin}             &  \textit{\nameref{tab:ur-re-01}} \\
    \end{tabular}
    \caption{Requirement SR-NF-03}
    \label{tab:sr-nf-03}
\end{table}

\begin{table}[H]
    \centering
    \begin{tabular}{l p{10cm}}
        \toprule
        \multicolumn{2}{c}{SR-NF-04} \\
        \toprule
        \textbf{Description}        & The ray-tracer should be implemented in Python \\
        \textbf{Necessity}          &  High \\
        \textbf{Priority}           &  High \\
        \textbf{Stability}          &  Stable \\
        \textbf{Verifiability}      & Verifiable \\
        \textbf{Origin}             &  \textit{\nameref{tab:ur-re-01}} \\
    \end{tabular}
    \caption{Requirement SR-NF-04}
    \label{tab:sr-nf-04}
\end{table}

\begin{table}[H]
    \centering
    \begin{tabular}{l p{10cm}}
        \toprule
        \multicolumn{2}{c}{SR-NF-05} \\
        \toprule
        \textbf{Description}        & The ray-tracer should be implemented in Go \\
        \textbf{Necessity}          &  High \\
        \textbf{Priority}           &  High \\
        \textbf{Stability}          &  Stable \\
        \textbf{Verifiability}      & Verifiable \\
        \textbf{Origin}             &  \textit{\nameref{tab:ur-re-01}} \\
    \end{tabular}
    \caption{Requirement SR-NF-05}
    \label{tab:sr-nf-05}
\end{table}

\begin{table}[H]
    \centering
    \begin{tabular}{l p{10cm}}
        \toprule
        \multicolumn{2}{c}{SR-NF-06} \\
        \toprule
        \textbf{Description}        & The ray-tracer's output should be stored in a file \\
        \textbf{Necessity}          &  High \\
        \textbf{Priority}           &  High \\
        \textbf{Stability}          &  Stable \\
        \textbf{Verifiability}      & Verifiable \\
        \textbf{Origin}             &  \textit{\nameref{tab:ur-ca-04}}, \textit{\nameref{tab:ur-ca-14}} \\
    \end{tabular}
    \caption{Requirement SR-NF-06}
    \label{tab:sr-nf-06}
\end{table}

\begin{table}[H]
    \centering
    \begin{tabular}{l p{10cm}}
        \toprule
        \multicolumn{2}{c}{SR-NF-07} \\
        \toprule
        \textbf{Description}        & The ray-tracer's randomness should not impact the structure of the scene \\
        \textbf{Necessity}          &  High \\
        \textbf{Priority}           &  High \\
        \textbf{Stability}          &  Stable \\
        \textbf{Verifiability}      & Verifiable \\
        \textbf{Origin}             & \textit{\nameref{tab:ur-ca-10}}, \textit{\nameref{tab:ur-ca-14}} \\
    \end{tabular}
    \caption{Requirement SR-NF-07}
    \label{tab:sr-nf-07}
\end{table}

\begin{table}[H]
    \centering
    \begin{tabular}{l p{10cm}}
        \toprule
        \multicolumn{2}{c}{SR-NF-08} \\
        \toprule
        \textbf{Description}        &  The system must implement different time-measuring systems for the different platforms\\
        \textbf{Necessity}          &  High \\
        \textbf{Priority}           &  High \\
        \textbf{Stability}          &  Stable \\
        \textbf{Verifiability}      & Verifiable \\
        \textbf{Origin}             & \textit{\nameref{tab:ur-ca-09}}, \textit{\nameref{tab:ur-ca-14}} \\
    \end{tabular}
    \caption{Requirement SR-NF-08}
    \label{tab:sr-nf-08}
\end{table}

\subsection{Functional Requirements}

\begin{table}[H]
    \centering
    \begin{tabular}{l p{10cm}}
        \toprule
        \multicolumn{2}{c}{SR-FN-01} \\
        \toprule
        \textbf{Description}        &  The system must measure the energy consumption of each of the benchmarks regardless of the implementation language. \\
        \textbf{Necessity}          &  High \\
        \textbf{Priority}           &  High \\
        \textbf{Stability}          &  Stable \\
        \textbf{Verifiability}      & Verifiable \\
        \textbf{Origin}             & \textit{\nameref{tab:ur-ca-05}}, \textit{\nameref{tab:ur-ca-09}}, \textit{\nameref{tab:ur-ca-16}}, \textit{\nameref{tab:ur-ca-17}} \\
    \end{tabular}
    \caption{Requirement SR-FN-01}
    \label{tab:sr-fn-01}
\end{table}

\begin{table}[H]
    \centering
    \begin{tabular}{l p{10cm}}
        \toprule
        \multicolumn{2}{c}{SR-FN-02} \\
        \toprule
        \textbf{Description}        &  The system must output the file where the energy consumption result is kept at the end of the execution. \\
        \textbf{Necessity}          &  High \\
        \textbf{Priority}           &  High \\
        \textbf{Stability}          &  Stable \\
        \textbf{Verifiability}      & Verifiable \\
        \textbf{Origin}             & \textit{\nameref{tab:ur-ca-09}}, \textit{\nameref{tab:ur-ca-14}} \\
    \end{tabular}
    \caption{Requirement SR-FN-02}
    \label{tab:sr-fn-02}
\end{table}

\begin{table}[H]
    \centering
    \begin{tabular}{l p{10cm}}
        \toprule
        \multicolumn{2}{c}{SR-FN-03} \\
        \toprule
        \textbf{Description}        &  The system must output the file where the execution time is kept at the end of the execution. \\
        \textbf{Necessity}          &  High \\
        \textbf{Priority}           &  High \\
        \textbf{Stability}          &  Stable \\
        \textbf{Verifiability}      & Verifiable \\
        \textbf{Origin}             & \textit{\nameref{tab:ur-ca-09}}, \textit{\nameref{tab:ur-ca-14}} \\
    \end{tabular}
    \caption{Requirement SR-FN-03}
    \label{tab:sr-fn-03}
\end{table}

\begin{table}[H]
    \centering
    \begin{tabular}{l p{10cm}}
        \toprule
        \multicolumn{2}{c}{SR-FN-04} \\
        \toprule
        \textbf{Description}        &  The system must reduce outlier results by executing the benchmarks multiple times. \\
        \textbf{Necessity}          &  High \\
        \textbf{Priority}           &  High \\
        \textbf{Stability}          &  Stable \\
        \textbf{Verifiability}      & Verifiable \\
        \textbf{Origin}             & \textit{\nameref{tab:ur-ca-12}}, \textit{\nameref{tab:ur-ca-16}}, \textit{\nameref{tab:ur-ca-17}} \\
    \end{tabular}
    \caption{Requirement SR-FN-04}
    \label{tab:sr-fn-04}
\end{table}

\begin{table}[H]
    \centering
    \begin{tabular}{l p{10cm}}
        \toprule
        \multicolumn{2}{c}{SR-FN-05} \\
        \toprule
        \textbf{Description}        &  The system must show the lines that are remaining to be processed. \\
        \textbf{Necessity}          &  High \\
        \textbf{Priority}           &  High \\
        \textbf{Stability}          &  Stable \\
        \textbf{Verifiability}      & Verifiable \\
        \textbf{Origin}             & \textit{\nameref{tab:ur-ca-15}} \\
    \end{tabular}
    \caption{Requirement SR-FN-05}
    \label{tab:sr-fn-05}
\end{table}

\begin{table}[H]
    \centering
    \begin{tabular}{l p{10cm}}
        \toprule
        \multicolumn{2}{c}{SR-FN-06} \\
        \toprule
        \textbf{Description}        &  The system must have a script to process MacOs Powermetrics results. \\
        \textbf{Necessity}          &  High \\
        \textbf{Priority}           &  High \\
        \textbf{Stability}          &  Stable \\
        \textbf{Verifiability}      & Verifiable \\
        \textbf{Origin}             & \textit{\nameref{tab:ur-ca-05}} \\
    \end{tabular}
    \caption{Requirement SR-FN-06}
    \label{tab:sr-fn-06}
\end{table}

\begin{table}[H]
    \centering
    \begin{tabular}{l p{10cm}}
        \toprule
        \multicolumn{2}{c}{SR-FN-07} \\
        \toprule
        \textbf{Description}        &  The system must have a file from which all benchmarks can be launched. \\
        \textbf{Necessity}          &  High \\
        \textbf{Priority}           &  High \\
        \textbf{Stability}          &  Stable \\
        \textbf{Verifiability}      & Verifiable \\
        \textbf{Origin}             & \textit{\nameref{tab:ur-ca-05}}, \textit{\nameref{tab:ur-ca-14}}, \textit{\nameref{tab:ur-ca-17}} \\
    \end{tabular}
    \caption{Requirement SR-FN-07}
    \label{tab:sr-fn-07}
\end{table}

\begin{table}[H]
    \centering
    \begin{tabular}{l p{10cm}}
        \toprule
        \multicolumn{2}{c}{SR-FN-08} \\
        \toprule
        \textbf{Description}        &  The system must have a file from which parameters for the simulation can be changed. \\
        \textbf{Necessity}          &  High \\
        \textbf{Priority}           &  High \\
        \textbf{Stability}          &  Stable \\
        \textbf{Verifiability}      & Verifiable \\
        \textbf{Origin}             & \textit{\nameref{tab:ur-ca-05}}, \textit{\nameref{tab:ur-ca-06}}, \textit{\nameref{tab:ur-ca-14}} \\
    \end{tabular}
    \caption{Requirement SR-FN-08}
    \label{tab:sr-fn-08}
\end{table}

\begin{table}[H]
    \centering
    \begin{tabular}{l p{10cm}}
        \toprule
        \multicolumn{2}{c}{SR-FN-09} \\
        \toprule
        \textbf{Description}        &  The system must have a utility that checks the diference between two output images. \\
        \textbf{Necessity}          &  High \\
        \textbf{Priority}           &  High \\
        \textbf{Stability}          &  Stable \\
        \textbf{Verifiability}      & Verifiable \\
        \textbf{Origin}             & \textit{\nameref{tab:ur-ca-04}} \\
    \end{tabular}
    \caption{Requirement SR-FN-09}
    \label{tab:sr-fn-09}
\end{table}


\begin{table}[H]
    \centering
    \begin{tabular}{l p{10cm}}
        \toprule
        \multicolumn{2}{c}{SR-FN-10} \\
        \toprule
        \textbf{Description}        &  The running of the benchamrks on any device must not interfere with the reading of the energy consumed by the program. \\
        \textbf{Necessity}          &  High \\
        \textbf{Priority}           &  High \\
        \textbf{Stability}          &  Stable \\
        \textbf{Verifiability}      & Verifiable \\
        \textbf{Origin}             & \textit{\nameref{tab:ur-ca-09}} \\
    \end{tabular}
    \caption{Requirement SR-FN-10}
    \label{tab:sr-fn-10}
\end{table}

\begin{table}[H]
    \centering
    \begin{tabular}{l p{10cm}}
        \toprule
        \multicolumn{2}{c}{SR-FN-11} \\
        \toprule
        \textbf{Description}        &  The running of the benchamrks on any device must not interfere with the reading of the time taken to execute the program. \\
        \textbf{Necessity}          &  High \\
        \textbf{Priority}           &  High \\
        \textbf{Stability}          &  Stable \\
        \textbf{Verifiability}      & Verifiable \\
        \textbf{Origin}             & \textit{\nameref{tab:ur-ca-09}} \\
    \end{tabular}
    \caption{Requirement SR-FN-11}
    \label{tab:sr-fn-11}
\end{table}

\section{Use Case}
\checkmark

\begin{figure}
    \hspace*{-2cm}
    \begin{tikzpicture}
        \begin{umlsystem}[] {Use Case Diagram for the Ray-Tracer Benchmarker} 
            \umlusecase[name=cpp,x=4,y=-2,width=2.2cm,fill=blue!20] {Run all C++ benchmarks}
            \umlusecase[name=cpp-sc,x=10,y=2,width=2.2cm,fill=blue!20] {Run C++ single-core}
            \umlusecase[name=cpp-mc,x=10,y=0,width=2.2cm,fill=blue!20] {Run C++ multi-core}
            \umlusecase[name=go,x=4,y=-4,width=2.2cm,fill=violet!20] {Run all Go benchmarks}
            \umlusecase[name=go-sc,x=10,y=-2,width=2.2cm,fill=violet!20] {Run Go single-core}
            \umlusecase[name=go-mc,x=10,y=-4,width=2.2cm,fill=violet!20] {Run Go multi-core}
            \umlusecase[name=python,x=4,y=-6,width=2.2cm,fill=orange!20] {Run all Python benchmarks}
            \umlusecase[name=python-sc,x=10,y=-7,width=2.2cm,fill=orange!20] {Run Python single-core}
            \umlusecase[name=python-mc,x=10,y=-9,width=2.2cm,fill=orange!20] {Run Python multi-core}
            \umlusecase[name=pypy,x=4,y=-8,width=2.2cm,fill=green!20] {Run all PyPy benchmarks}
            \umlusecase[name=pypy-sc,x=10,y=-11,width=2.2cm,fill=green!20] {Run PyPy single-core}
            \umlusecase[name=pypy-mc,x=10,y=-13,width=2.2cm,fill=green!20] {Run PyPy multi-core}
            
            \umlusecase[name=all,y=-5,width=2.2cm] {All benchmarks}
            \umlusecase[name=all-sc,x=12,y=5.5,width=2.2cm] {All single-core benchmarks}
            \umlusecase[name=all-mc,x=12,y=-16.5,width=2.2cm] {All multicore benchmarks}

            \umlusecase[name=view-image,y=-10,width=2.2cm] {View image}

            

        \end{umlsystem}
        \node [above] at (current bounding box.north) {Ray-Tracer Benchmarker};

        \umlactor[y=-5,x=-3.2] {User}

        \umlassoc{User}{cpp}
        \umlassoc{User}{cpp-sc}
        \umlassoc{User}{cpp-mc}
        \umlassoc{User}{go}
        \umlassoc{User}{go-sc}
        \umlassoc{User}{go-mc}
        \umlassoc{User}{python}
        \umlassoc{User}{python-sc}
        \umlassoc{User}{python-mc}
        \umlassoc{User}{pypy}
        \umlassoc{User}{pypy-sc}
        \umlassoc{User}{pypy-mc}
        \umlassoc{User}{all}
        \umlassoc{User}{all-sc}
        \umlassoc{User}{all-mc}
        \umlassoc{User}{view-image}



        % CPP extends single and multi
        \draw [tikzuml dependency style] (cpp) -- node[above] {\scriptsize $\ll \text{extend} \gg$} (cpp-sc);
        \draw [tikzuml dependency style] (cpp) -- node[above] {\scriptsize $\ll \text{extend} \gg$} (cpp-mc);
        % Go 
        \draw [tikzuml dependency style] (go) -- node[below] {\scriptsize $\ll \text{extend} \gg$} (go-sc);
        \draw [tikzuml dependency style] (go) -- node[above] {\scriptsize $\ll \text{extend} \gg$} (go-mc);
        % Python
        \draw [tikzuml dependency style] (python) -- node[above] {\scriptsize $\ll \text{extend} \gg$} (python-sc);
        \draw [tikzuml dependency style] (python) -- node[above] {\scriptsize $\ll \text{extend} \gg$} (python-mc);
        % PyPy
        \draw [tikzuml dependency style] (pypy) -- node[above] {\scriptsize $\ll \text{extend} \gg$} (pypy-sc);
        \draw [tikzuml dependency style] (pypy) -- node[below] {\scriptsize $\ll \text{extend} \gg$} (pypy-mc);
        % PyPy
        \draw [tikzuml dependency style] (all) to[bend right=20] node[midway, below] {\scriptsize $\ll \text{extend} \gg$} (all-mc);
        \draw [tikzuml dependency style] (all) to[bend left=20] node[midway, above] {\scriptsize $\ll \text{extend} \gg$} (all-sc);


        % Single
        \draw [tikzuml dependency style] (all-sc) to[bend left=0] node[midway, above] { \scriptsize $\ll \text{extend} \gg$} (cpp-sc);
        \draw [tikzuml dependency style] (all-sc) to[bend left=35] node[midway, above] {\scriptsize $\ll \text{extend} \gg$} (go-sc);
        \draw [tikzuml dependency style] (all-sc) to[bend left=26] node[midway, above] {\scriptsize $\ll \text{extend} \gg$} (python-sc);
        \draw [tikzuml dependency style] (all-sc) to[bend left=38] node[midway, above] {\scriptsize $\ll \text{extend} \gg$} (pypy-sc);

        % Multi-core
        \draw [tikzuml dependency style] (all-mc) to[bend right=38] node[midway, above] {\scriptsize $\ll \text{extend} \gg$} (cpp-mc);
        \draw [tikzuml dependency style] (all-mc) to[bend right=26] node[midway, above] {\scriptsize $\ll \text{extend} \gg$} (go-mc);
        \draw [tikzuml dependency style] (all-mc) to[bend right=35] node[midway, above] {\scriptsize $\ll \text{extend} \gg$} (python-mc);
        \draw [tikzuml dependency style] (all-mc) to[bend right=0] node[midway, above] {\scriptsize $\ll \text{extend} \gg$} (pypy-mc);
        
    \end{tikzpicture}
    \caption{Use Case Diagram for the Ray-Tracer Benchmarker}
    \label{fig:use-case}
\end{figure}


\newpage

\begin{landscape}
\begin{figure}
    \hspace*{-1.5cm}
    \centering
    \begin{tikzpicture}
        \umlactor[name=user,x=-4,y=-14,] {User}

        \begin{umlsystem}[x=-2,y=-12]{Ray-Tracer Benchmarker}

            \umlusecase[x =  0, y =  0, name=runall]{UC-01: Run all benchmarks}
            \umlusecase[x =  6, y =  0, width=2cm, name=runcpp,    fill=blue!20]{UC-02: Run C++ benchmarks}
            \umlusecase[x =  6, y = -3, width=2cm, name=rungo,     fill=violet!20]{UC-03: Run Go benchmarks}
            \umlusecase[x =  6, y = -6, width=2cm, name=runpy,     fill=orange!20]{UC-04: Run Python benchmarks}
            \umlusecase[x =  6, y = -9, width=2cm, name=runpypy,   fill=green!20]{UC-05: Run PyPy benchmarks}

            \umlusecase[x = 14, y =  0, width = 2.5cm, fill = gray!10, name=single]{UC-06:\\Run single-core}
            \umlusecase[x = 14, y = -3, width = 2.5cm, fill = gray!10, name=multi]{UC-07:\\Run multicore}

            \umlusecase[width=3cm, x = 20, y =  2, name=cppsingle,fill=blue!20]{UC-02.1: C++ single-core}
            \umlusecase[width=3cm, x = 20, y = 0.25, name=cppmulti,fill=blue!20]{UC-02.2: C++ multi-core}
            \umlusecase[width=3cm, x = 20, y = -1.5, name=gosingle,fill=violet!20]{UC-03.1: Go single-core}
            \umlusecase[width=3cm, x = 20, y = -3.25, name=gomulti,fill=violet!20]{UC-03.2: Go multi-core}
            \umlusecase[width=3cm, x = 20, y = -5, name=pysingle,fill=orange!20]{UC-04.1: Python single-core}
            \umlusecase[width=3cm, x = 20, y = -6.75, name=pymulti,fill=orange!20]{UC-04.2: Python multi-core}
            \umlusecase[width=3cm, x = 20, y = -8.5, name=pypysingle,fill=green!20]{UC-05.1: PyPy single-core}
            \umlusecase[width=3cm, x = 20, y = -10.25, name=pypymulti,fill=green!20]{UC-05.2: PyPy multi-core}

            \umlassoc{User}{runall}
            \umlassoc{User}{runcpp}
            \umlassoc{User}{rungo}
            \umlassoc{User}{runpy}
            \umlassoc{User}{runpypy}

            % Include relationships from runall - positioned at top and bottom
            \draw [tikzuml dependency style] (runall) to[bend left=16] node[above] {$\ll \text{extend} \gg$} (single);
            % \umlextend[anchor1=45,anchor2=-225,pos=0.7]{runall}{single}
            \umlextend[anchor1=315,anchor2=135,pos=0.2]{runall}{multi}

            % Include relationships to single-core - staggered positions
            \umlinclude[anchor1=0,anchor2=180,pos=0.6]{runcpp}{single}
            \umlinclude[anchor1=45,anchor2=210,pos=0.4]{rungo}{single}
            \umlinclude[anchor1=60,anchor2=220,pos=0.6]{runpy}{single}
            \umlinclude[anchor1=70,anchor2=230,pos=0.4]{runpypy}{single}

            % Include relationships to multi-core - staggered positions
            \umlinclude[anchor1=-30,anchor2=180,pos=0.4]{runcpp}{multi}
            \umlinclude[anchor1=330,anchor2=190,pos=0.6]{rungo}{multi}
            \umlinclude[anchor1=-60,anchor2=210,pos=0.4]{runpy}{multi}
            \umlinclude[anchor1=270,anchor2=230,pos=0.3]{runpypy}{multi}

            % Inheritance relationships
            \umlinherit[anchor2=180]{single}{cppsingle}
            \umlinherit[anchor2=180]{single}{gosingle}
            \umlinherit[anchor2=180]{single}{pysingle}
            \umlinherit[anchor2=180]{single}{pypysingle}

            \umlinherit[anchor2=180]{multi}{cppmulti}
            \umlinherit[anchor2=180]{multi}{gomulti}
            \umlinherit[anchor2=180]{multi}{pymulti}
            \umlinherit[anchor2=180]{multi}{pypymulti}
        \end{umlsystem}

    \end{tikzpicture}
    \caption{Cleaned-up use-case diagram for the Ray-Tracer Benchmarker}
    \label{fig:use-case-v2}
\end{figure}
\end{landscape}



Use case template:

\begin{table}[H]
    \centering
    \begin{tabular}{l p{10cm}}
        \toprule
        \multicolumn{2}{c}{\textbf{ID: UC-xx}} \\
        \toprule
        \textbf{Name}               &  The name of the use case inside de diagram. \\
        \textbf{Actors}             &  User, System \\
        \textbf{Objective}          &  Brief description of the goal of the use case. \\
        \textbf{Description}        &  Steps the actor has to entail. \\
        \textbf{Preconditions}      &  The user has the system installed and configured. \\
        \textbf{Postconditions}     &  The benchmarks are executed and the results are stored. \\
    \end{tabular}
    \caption{Use Case UC-xx}
    \label{tab:uc-xx}
\end{table}

Start of use cases:
\begin{table}[H]
    \centering
    \begin{tabular}{l p{10cm}}
        \toprule
        \multicolumn{2}{c}{\textbf{ID: UC-01}} \\
        \toprule
        \textbf{Name}                         &  Run all benchamrks. \\
        \textbf{Actors}                       &  User \\
        \textbf{Objective}                    &  Runing all benchmarks, multi-core and single-core versions of all the languages implemented. \\
        \multirow{1}{*}{\textbf{Descripción}} & \textsl{1.} The user writes the command to run the benchmarks: \texttt{make all}.\\
                                              & \textsl{2.} The user defines the number of cores to be used for the multi-core benchmarks, CORES=<num>.\\
                                              & \textsl{3.} The user defines the platform to be used for the benchmarks <(SERVER / MACOS / RPI)>=True.\\
                                              & \textsl{4.} Finish the execution of the benchmarks.\\
                                              & \textsl{5.} Open the resulting files.\\ 
        \textbf{Preconditions}                &  N/A \\
        \textbf{Postconditions}               &  The benchmarks are executed and the results are stored and those files are outputed. \\
    \end{tabular}
    \caption{Use Case UC-01}
    \label{tab:uc-01}
\end{table}

\begin{table}[H]
    \centering
    \begin{tabular}{l p{10cm}}
        \toprule
        \multicolumn{2}{c}{\textbf{ID: UC-02}} \\
        \toprule
        \textbf{Name}                         &  Run C++ Benchmarks. \\
        \textbf{Actors}                       &  User \\
        \textbf{Objective}                    &  Runing all C++ implementation of the benchmarks, multi-core and single-core versions. \\
        \multirow{1}{*}{\textbf{Descripción}} & \textsl{1.} The user writes the command to run the benchmarks: \texttt{make cpp cpp-single}.\\
                                              & \textsl{2.} The user defines the number of cores to be used for the multi-core benchmarks, CORES=<num>.\\
                                              & \textsl{3.} The user defines the platform to be used for the benchmarks <(SERVER / MACOS / RPI)>=True.\\
                                              & \textsl{4.} Finish the execution of the benchmarks.\\
                                              & \textsl{5.} Open the resulting files.\\ 
        \textbf{Preconditions}                &  N/A \\
        \textbf{Postconditions}               &  The C++ benchmarks are executed and the results are stored and those files are outputed. \\
    \end{tabular}
    \caption{Use Case UC-02}
    \label{tab:uc-single-core 02}
\end{table}

\begin{table}[H]
    \centering
    \begin{tabular}{l p{10cm}}
        \toprule
        \multicolumn{2}{c}{\textbf{ID: UC-02.1}} \\
        \toprule
        \textbf{Name}                         &  Run C++ Benchmarks. \\
        \textbf{Actors}                       &  User \\
        \textbf{Objective}                    &  Runing the C++ implementation of the single-core benchmarks. \\
        \multirow{1}{*}{\textbf{Descripción}} & \textsl{1.} The user writes the command to run the benchmarks: \texttt{make cpp-single}.\\
                                              & \textsl{2.} The user defines the platform to be used for the benchmarks <(SERVER / MACOS / RPI)>=True.\\
                                              & \textsl{3.} Finish the execution of the benchmarks.\\
                                              & \textsl{4.} Open the resulting files.\\ 
        \textbf{Preconditions}                &  N/A \\
        \textbf{Postconditions}               &  The C++ single-core benchmark is executed and the results are stored and those files are outputed. \\
    \end{tabular}
    \caption{Use Case UC-02.1}
    \label{tab:uc-02.1}
\end{table}

\begin{table}[H]
    \centering
    \begin{tabular}{l p{10cm}}
        \toprule
        \multicolumn{2}{c}{\textbf{ID: UC-02.2}} \\
        \toprule
        \textbf{Name}                         &  Run C++ Benchmarks. \\
        \textbf{Actors}                       &  User \\
        \textbf{Objective}                    &  Runing the C++ implementation of the multi-core benchmarks. \\
        \multirow{1}{*}{\textbf{Descripción}} & \textsl{1.} The user writes the command to run the benchmarks: \texttt{make cpp}.\\
                                              & \textsl{2.} The user defines the number of cores to be used for the multi-core benchmarks, CORES=<num>.\\
                                              & \textsl{3.} The user defines the platform to be used for the benchmarks <(SERVER / MACOS / RPI)>=True.\\
                                              & \textsl{4.} Finish the execution of the benchmarks.\\
                                              & \textsl{5.} Open the resulting files.\\ 
        \textbf{Preconditions}                &  N/A \\
        \textbf{Postconditions}               &  The C++ multi-core benchmark is executed and the results are stored and those files are outputed. \\
    \end{tabular}
    \caption{Use Case UC-02.2}
    \label{tab:uc-02.2}
\end{table}



\begin{table}[H]
    \centering
    \begin{tabular}{l p{10cm}}
        \toprule
        \multicolumn{2}{c}{\textbf{ID: UC-03}} \\
        \toprule
        \textbf{Name}                         &  Run Go Benchmarks. \\
        \textbf{Actors}                       &  User \\
        \textbf{Objective}                    &  Runing all Go implementation of the benchmarks, multi-core and single-core versions. \\
        \multirow{1}{*}{\textbf{Descripción}} & \textsl{1.} The user writes the command to run the benchmarks: \texttt{make go go-single}.\\
                                              & \textsl{2.} The user defines the number of cores to be used for the multi-core benchmarks, CORES=<num>.\\
                                              & \textsl{3.} The user defines the platform to be used for the benchmarks <(SERVER / MACOS / RPI)>=True.\\
                                              & \textsl{4.} Finish the execution of the benchmarks.\\
                                              & \textsl{5.} Open the resulting files.\\ 
        \textbf{Preconditions}                &  N/A \\
        \textbf{Postconditions}               &  The Go benchmarks are executed and the results are stored and those files are outputed. \\
    \end{tabular}
    \caption{Use Case UC-03}
    \label{tab:uc-03}
\end{table}


\begin{table}[H]
    \centering
    \begin{tabular}{l p{10cm}}
        \toprule
        \multicolumn{2}{c}{\textbf{ID: UC-03.1}} \\
        \toprule
        \textbf{Name}                         &  Run Go Benchmarks. \\
        \textbf{Actors}                       &  User \\
        \textbf{Objective}                    &  Runing the Go implementation of the single-core benchmarks. \\
        \multirow{1}{*}{\textbf{Descripción}} & \textsl{1.} The user writes the command to run the benchmarks: \texttt{make go-single}.\\
                                              & \textsl{2.} The user defines the platform to be used for the benchmarks <(SERVER / MACOS / RPI)>=True.\\
                                              & \textsl{3.} Finish the execution of the benchmarks.\\
                                              & \textsl{4.} Open the resulting files.\\ 
        \textbf{Preconditions}                &  N/A \\
        \textbf{Postconditions}               &  The Go single-core benchmark is executed and the results are stored and those files are outputed. \\
    \end{tabular}
    \caption{Use Case UC-03.1}
    \label{tab:uc-03.1}
\end{table}

\begin{table}[H]
    \centering
    \begin{tabular}{l p{10cm}}
        \toprule
        \multicolumn{2}{c}{\textbf{ID: UC-03.2}} \\
        \toprule
        \textbf{Name}                         &  Run Go Benchmarks. \\
        \textbf{Actors}                       &  User \\
        \textbf{Objective}                    &  Runing the Go implementation of the multi-core benchmarks. \\
        \multirow{1}{*}{\textbf{Descripción}} & \textsl{1.} The user writes the command to run the benchmarks: \texttt{make go}.\\
                                              & \textsl{2.} The user defines the number of cores to be used for the multi-core benchmarks, CORES=<num>.\\
                                              & \textsl{3.} The user defines the platform to be used for the benchmarks <(SERVER / MACOS / RPI)>=True.\\
                                              & \textsl{4.} Finish the execution of the benchmarks.\\
                                              & \textsl{5.} Open the resulting files.\\ 
        \textbf{Preconditions}                &  N/A \\
        \textbf{Postconditions}               &  The Go multi-core benchmark is executed and the results are stored and those files are outputed. \\
    \end{tabular}
    \caption{Use Case UC-03.2}
    \label{tab:uc-03.2}
\end{table}


\begin{table}[H]
    \centering
    \begin{tabular}{l p{10cm}}
        \toprule
        \multicolumn{2}{c}{\textbf{ID: UC-04}} \\
        \toprule
        \textbf{Name}                         &  Run Python Benchmarks. \\
        \textbf{Actors}                       &  User \\
        \textbf{Objective}                    &  Runing all Python implementation of the benchmarks, multi-core and single-core versions. \\
        \multirow{1}{*}{\textbf{Descripción}} & \textsl{1.} The user writes the command to run the benchmarks: \texttt{make python python-single}.\\
                                              & \textsl{2.} The user defines the number of cores to be used for the multi-core benchmarks, CORES=<num>.\\
                                              & \textsl{3.} The user defines the platform to be used for the benchmarks <(SERVER / MACOS / RPI)>=True.\\
                                              & \textsl{4.} Finish the execution of the benchmarks.\\
                                              & \textsl{5.} Open the resulting files.\\ 
        \textbf{Preconditions}                &  N/A \\
        \textbf{Postconditions}               &  The Python benchmarks are executed and the results are stored and those files are outputed. \\
    \end{tabular}
    \caption{Use Case UC-04}
    \label{tab:uc-04}
\end{table}


\begin{table}[H]
    \centering
    \begin{tabular}{l p{10cm}}
        \toprule
        \multicolumn{2}{c}{\textbf{ID: UC-04.1}} \\
        \toprule
        \textbf{Name}                         &  Run Python Benchmarks. \\
        \textbf{Actors}                       &  User \\
        \textbf{Objective}                    &  Runing the Python implementation of the single-core benchmarks. \\
        \multirow{1}{*}{\textbf{Descripción}} & \textsl{1.} The user writes the command to run the benchmarks: \texttt{make python-single}.\\
                                              & \textsl{2.} The user defines the platform to be used for the benchmarks <(SERVER / MACOS / RPI)>=True.\\
                                              & \textsl{3.} Finish the execution of the benchmarks.\\
                                              & \textsl{4.} Open the resulting files.\\ 
        \textbf{Preconditions}                &  N/A \\
        \textbf{Postconditions}               &  The Python single-core benchmark is executed and the results are stored and those files are outputed. \\
    \end{tabular}
    \caption{Use Case UC-04.1}
    \label{tab:uc-04.1}
\end{table}

\begin{table}[H]
    \centering
    \begin{tabular}{l p{10cm}}
        \toprule
        \multicolumn{2}{c}{\textbf{ID: UC-04.2}} \\
        \toprule
        \textbf{Name}                         &  Run Python Benchmarks. \\
        \textbf{Actors}                       &  User \\
        \textbf{Objective}                    &  Runing the Python implementation of the multi-core benchmarks. \\
        \multirow{1}{*}{\textbf{Descripción}} & \textsl{1.} The user writes the command to run the benchmarks: \texttt{make python}.\\
                                              & \textsl{2.} The user defines the number of cores to be used for the multi-core benchmarks, CORES=<num>.\\
                                              & \textsl{3.} The user defines the platform to be used for the benchmarks <(SERVER / MACOS / RPI)>=True.\\
                                              & \textsl{4.} Finish the execution of the benchmarks.\\
                                              & \textsl{5.} Open the resulting files.\\ 
        \textbf{Preconditions}                &  N/A \\
        \textbf{Postconditions}               &  The Python multi-core benchmark is executed and the results are stored and those files are outputed. \\
    \end{tabular}
    \caption{Use Case UC-04.2}
    \label{tab:uc-04.2}
\end{table}


\begin{table}[H]
    \centering
    \begin{tabular}{l p{10cm}}
        \toprule
        \multicolumn{2}{c}{\textbf{ID: UC-05}} \\
        \toprule
        \textbf{Name}                         &  Run Pypy Benchmarks. \\
        \textbf{Actors}                       &  User \\
        \textbf{Objective}                    &  Runing all Pypy implementation of the benchmarks, multi-core and single-core versions. \\
        \multirow{1}{*}{\textbf{Descripción}} & \textsl{1.} The user writes the command to run the benchmarks: \texttt{make pypy pypy-single}.\\
                                              & \textsl{2.} The user defines the number of cores to be used for the multi-core benchmarks, CORES=<num>.\\
                                              & \textsl{3.} The user defines the platform to be used for the benchmarks <(SERVER / MACOS / RPI)>=True.\\
                                              & \textsl{4.} Finish the execution of the benchmarks.\\
                                              & \textsl{5.} Open the resulting files.\\ 
        \textbf{Preconditions}                &  N/A \\
        \textbf{Postconditions}               &  The Pypy benchmarks are executed and the results are stored and those files are outputed. \\
    \end{tabular}
    \caption{Use Case UC-05}
    \label{tab:uc-05}
\end{table}


\begin{table}[H]
    \centering
    \begin{tabular}{l p{10cm}}
        \toprule
        \multicolumn{2}{c}{\textbf{ID: UC-05.1}} \\
        \toprule
        \textbf{Name}                         &  Run Pypy Benchmarks. \\
        \textbf{Actors}                       &  User \\
        \textbf{Objective}                    &  Runing the Pypy implementation of the single-core benchmarks. \\
        \multirow{1}{*}{\textbf{Descripción}} & \textsl{1.} The user writes the command to run the benchmarks: \texttt{make pypy-single}.\\
                                              & \textsl{2.} The user defines the platform to be used for the benchmarks <(SERVER / MACOS / RPI)>=True.\\
                                              & \textsl{3.} Finish the execution of the benchmarks.\\
                                              & \textsl{4.} Open the resulting files.\\ 
        \textbf{Preconditions}                &  N/A \\
        \textbf{Postconditions}               &  The Pypy single-core benchmark is executed and the results are stored and those files are outputed. \\
    \end{tabular}
    \caption{Use Case UC-05.1}
    \label{tab:uc-05.1}
\end{table}

\begin{table}[H]
    \centering
    \begin{tabular}{l p{10cm}}
        \toprule
        \multicolumn{2}{c}{\textbf{ID: UC-05.2}} \\
        \toprule
        \textbf{Name}                         &  Run Pypy Benchmarks. \\
        \textbf{Actors}                       &  User \\
        \textbf{Objective}                    &  Runing the Pypy implementation of the multi-core benchmarks. \\
        \multirow{1}{*}{\textbf{Descripción}} & \textsl{1.} The user writes the command to run the benchmarks: \texttt{make pypy}.\\
                                              & \textsl{2.} The user defines the number of cores to be used for the multi-core benchmarks, CORES=<num>.\\
                                              & \textsl{3.} The user defines the platform to be used for the benchmarks <(SERVER / MACOS / RPI)>=True.\\
                                              & \textsl{4.} Finish the execution of the benchmarks.\\
                                              & \textsl{5.} Open the resulting files.\\ 
        \textbf{Preconditions}                &  N/A \\
        \textbf{Postconditions}               &  The Pypy multi-core benchmark is executed and the results are stored and those files are outputed. \\
    \end{tabular}
    \caption{Use Case UC-05.2}
    \label{tab:uc-05.2}
\end{table}



\begin{table}[H]
    \centering
    \begin{tabular}{l p{10cm}}
        \toprule
        \multicolumn{2}{c}{\textbf{ID: UC-06}} \\
        \toprule
        \textbf{Name}                         &  Run Single core Benchmarks. \\
        \textbf{Actors}                       &  User \\
        \textbf{Objective}                    &  Runing all Pypy implementation of the single-core version of the benchmarks. \\
        \multirow{1}{*}{\textbf{Descripción}} & \textsl{1.} The user writes the command to run the benchmarks: \texttt{make all-single}.\\
                                              & \textsl{2.} The user defines the platform to be used for the benchmarks <(SERVER / MACOS / RPI)>=True.\\
                                              & \textsl{3.} Finish the execution of the benchmarks.\\
                                              & \textsl{4.} Open the resulting files.\\ 
        \textbf{Preconditions}                &  N/A \\
        \textbf{Postconditions}               &  The single-core implementation of the benchmarks are executed and the results are stored and those files are outputed. \\
    \end{tabular}
    \caption{Use Case UC-06}
    \label{tab:uc-06}
\end{table}

\begin{table}[H]
    \centering
    \begin{tabular}{l p{10cm}}
        \toprule
        \multicolumn{2}{c}{\textbf{ID: UC-07}} \\
        \toprule
        \textbf{Name}                         &  Run multi core Benchmarks. \\
        \textbf{Actors}                       &  User \\
        \textbf{Objective}                    &  Runing all Pypy implementation of the multi-core version of the benchmarks. \\
        \multirow{1}{*}{\textbf{Descripción}} & \textsl{1.} The user writes the command to run the benchmarks: \texttt{make multi}.\\
                                              & \textsl{2.} The user defines the number of cores to be used for the multi-core benchmarks, CORES=<num>.\\
                                              & \textsl{3.} The user defines the platform to be used for the benchmarks <(SERVER / MACOS / RPI)>=True.\\
                                              & \textsl{4.} Finish the execution of the benchmarks.\\
                                              & \textsl{5.} Open the resulting files.\\ 
        \textbf{Preconditions}                &  N/A \\
        \textbf{Postconditions}               &  The multi-core implementation of the benchmarks are executed and the results are stored and those files are outputed. \\
    \end{tabular}
    \caption{Use Case UC-07}
    \label{tab:uc-07}
\end{table}



\section{Traceability}
