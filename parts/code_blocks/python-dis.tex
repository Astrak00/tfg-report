\lstdefinestyle{pythonstyle}{
    language=Python,
    backgroundcolor=\color{black!5},   % light grey background
    commentstyle=\color{green!40!black},
    keywordstyle=\color{blue},
    stringstyle=\color{purple},
    numberstyle=\tiny\color{gray},
    basicstyle=\ttfamily\small,        % use a typewriter font
    breakatwhitespace=false,         
    breaklines=true,                 
    captionpos=b,                    % puts the caption at the bottom
    keepspaces=true,                 % respects spaces, crucial for Python
    numbers=left,                    
    numbersep=5pt,                   % space between line numbers and code
    showspaces=false,                
    showstringspaces=false,
    showtabs=false,
    frame=single,                    % adds a frame around the code
    rulecolor=\color{black},
    tabsize=2
}

% Apply this style to all listings environments globally
\lstset{style=estilo}

\section*{A Concrete Example with \texttt{dis}}

% --- PYTHON CODE LISTING ---
\begin{lstlisting}[language=Python, caption={Python code demonstrating the \texttt{dis} module.}, label={lst:dis_example}
]
import dis

def simple_math(a):
    x = a + 10
    return x

# Use the disassembler to inspect the function's bytecode
dis.dis(simple_math)
\end{lstlisting}

Let's see this in action. The \texttt{dis} module is a `disassembler' that 
shows you the bytecode for a piece of Python code. The script in \autoref{lst:dis_example} defines a simple function and then uses \texttt{dis} to inspect it.

\begin{lstlisting}[
    language={}, 
    caption={Bytecode output generated by the \texttt{dis.dis} function.},
    label={lst:dis_output}]
  4           0 LOAD_FAST                0 (a)
              2 LOAD_CONST               1 (10)
              4 BINARY_ADD
              6 STORE_FAST               1 (x)

  5           8 LOAD_FAST                1 (x)
             10 RETURN_VALUE
\end{lstlisting}

The output of this script, shown in \autoref{lst:dis_output}, reveals the low-level bytecode instructions that the Python Virtual Machine will execute.

