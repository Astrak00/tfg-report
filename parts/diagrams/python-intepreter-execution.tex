\tikzset{
  block/.style = {
    rectangle,
    draw,
    fill=blue!20,
    text width=10em,
    text centered,
    rounded corners,
    minimum height=2em
  },
  io/.style = {
    trapezium,
    trapezium left angle=70,
    trapezium right angle=110,
    draw,
    fill=orange!20,
    text centered,
    minimum height=1em
  },
  line/.style = {draw, -latex'}
}

\begin{figure}
  \centering
  % Define node distance globally for the picture
  \begin{tikzpicture}[node distance = 1cm and 1.5cm]
    % Place nodes using the positioning library syntax
    \node [io] (bytecode) {Bytecode Sequence};

    \node [block, below = of bytecode] (fetch)
      {a. Fetch next bytecode instruction};

    \node [block, below = of fetch] (decode)
      {b. Determine instruction's meaning (Decode)};

    \node [block, below = of decode] (execute)
      {c. Execute the instruction};

    \node [block, below = of execute] (move)
      {d. Move to next instruction};

    \node [io, right = of execute, xshift=1.5cm, text width=5em] (data)
      {Manipulate Call Stack \& Data Heap};

    % Draw arrows (your path commands were already perfect)
    \path [line] (bytecode) -- (fetch);
    \path [line] (fetch) -- (decode);
    \path [line] (decode) -- (execute);
    \path [line] (execute) -- (move);

    % Arrow showing interaction with data/stack
    \path [draw, <->, -latex'] (execute.east) --
      node[midway, above, text width=2cm, text centered] {Read/Write}
      (data.west);

    % The loop back arrow using orthogonal paths
    \path [line] (move.west) -- ++(-1.5, 0) |- (fetch);

  \end{tikzpicture}
  \caption{Python’s Execution loop}
  \label{fig:python-loop}
\end{figure}
