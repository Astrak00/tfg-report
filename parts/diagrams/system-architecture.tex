\tikzset{
    auto,
    node distance=1.5cm and 2.5cm,
    block/.style={
    rectangle,
    draw,
    fill=blue!20,
    text width=8em,
    text centered,
    rounded corners,
    minimum height=3em,
    drop shadow
    },
    io/.style={
    trapezium,
    trapezium left angle=70,
    trapezium right angle=110,
    draw,
    fill=gray!20,
    text width=7em,
    text centered,
    minimum height=3em,
    drop shadow
    },
    class/.style={
    rectangle,
    draw,
    fill=green!20,
    text width=7em,
    text centered,
    rounded corners,
    minimum height=3em,
    drop shadow
    },
    baseclass/.style={
    class,
    fill=green!40,
    font=\bfseries
    },
    datastruct/.style={
    rectangle,
    draw,
    fill=orange!20,
    text width=6em,
    text centered,
    rounded corners,
    minimum height=2.5em,
    drop shadow
    },
    line/.style={-latex, thick}
}

\begin{landscape} 
    \section{System Architecture} 

        \begin{tikzpicture} 
            % Nodes - Repositioned for better spacing
            \node[block, text width=10em] (main) {program entrypoint (main)};
            \node[io, above=1cm of main] (makefile) {Makefile};
            \node[io, left=2.5cm of main] (scene_data) {sphere\_data.txt};

            % Core classes - better spacing
            \node[class, below left=2cm and 1cm of main] (camera) {camera};
            \node[class, below right=2cm and 0cm of main] (hittable_list) {hittable\_list (world)};
            \node[baseclass, right=3cm of hittable_list] (hittable) {hittable};

            % Material hierarchy - moved down and spread out
            \node[class, below=2.5cm of hittable] (sphere) {sphere};
            \node[baseclass, left=3.5cm of sphere] (material) {material};

            % Data structures - repositioned to avoid conflicts
            \node[class, below=1.5cm of camera] (image) {Image};
            \node[datastruct, below right=0.75cm and 2cm of camera] (vec3) {vec3 / color / point3};
            \node[io, below=1cm of image] (output_ppm) {output.ppm};

            % Arrows - rerouted to avoid overlaps
            \draw[line] (makefile) -- node[midway, right] {builds \& runs} (main);
            \draw[line] (scene_data) -- node[midway, above] {reads scene} (main);

            % Main to components - using better angles
            \draw[line] (main) -- node[midway, left] {configures} (camera);
            \draw[line] (main) -- node[midway, right] {populates} (hittable_list);

            % Camera connections - routed around components
            \draw[line] (camera) -- node[midway, left] {writes pixels} (image);
            \draw[line] (image) .. controls +(south:2) and +(north:2) .. node[midway, left] {writes file} (output_ppm);

            % World interaction - curved to avoid conflicts
            \draw[line] (camera) .. controls +(east:1.5) and +(west:2) .. node[midway, above] {world.hit()} (hittable_list);

            % Inheritance relationships
            \draw[line] (hittable_list) -- node[midway, above] {contains} (hittable);
            \draw[line,dashed] (sphere) -- node[midway, right] {inherits} (hittable);
            \draw[line] (sphere) .. controls +(west:2) and +(east:2) .. node[midway, above] {has a} (material);

            % Data structure connections - routed to avoid overlaps
            \draw[line] (camera) -- (vec3);
            \draw[line] (sphere) -- (vec3);
        \end{tikzpicture}

        \captionof{figure}{System Architecture Diagram for the Ray Tracer.}\label{fig:system-architecture}
\end{landscape}