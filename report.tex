\documentclass[en]{uc3mthesisIEEE}


\usepackage{import}
\usepackage{enumitem}  % control item separation -> \begin{itemize}[nosep]
\usepackage{lipsum}  % dummy text
\usepackage{placeins}  % \FloatBarrier -> prevents figures and tables from passing that point
\usepackage{hyperref} % For referencing url
\usepackage{listings} % For code blocks
\usepackage{xcolor} % For using custom colors
\usepackage{tabularx} % For auto-spacing columns
\usepackage{booktabs} 
\usepackage{floatrow}
\usepackage{pgfplots}
\usepackage{amsmath}
\usepackage{tikz-uml}
\usepackage{pdflscape}
\tikzumlset{fill usecase=white}
\pgfplotsset{compat=1.18}

\usepackage{tikz} % For drawing flowcharts and graphs
\usetikzlibrary{
    positioning,
    shapes.geometric,
    arrows, % Using the compatible 'arrows' library
    arrows.meta,
    fit,
    backgrounds
}

\usepackage{array}


\usepackage{mymacros}  % report-specific macros
\usepackage{glossaries}
% \makenoidxglossaries

% silence ht warnings
\usepackage{silence}
\WarningFilter{latex}{`h' float specifier changed to `ht'}


\graphicspath{{img/}}  % Images folder


% REFERENCES
\addbibresource{references.bib}  % bibliography file
\import{}{glossary.tex}  % glossary file


%	DOCUMENT

% setup
\degree{Bachelor's degree in Computer Science and Engineering}
\title{Evaluating performance and energy impact of programming languages}
% \shorttitle{I like big butts}
\author{Eduardo Alarcón Navarro}
\advisors{Jose Daniel García Sánchez}
\place{Leganés, Madrid, Spain}
\date{September 2025}

\begin{document}

  % COVER
  \makecover


  % EPIGRAPH
  \makeepigraph
    {To reach the moon, you should aim for the stars.}  % quote
    {}  % author
    {}  % source


  % ACKNOWLEDGEMENTS
  \begin{acknowledgements}
    % 
    I would like to thank my family, for their continuous support and encouragement, specially this last year, where it has been the hardest for all of us. Not only they have provided me with the best education possible, financially and emotionally, but they have also taught me the importance of hard work and dedication. 

    I would like to thank my parents, my father for always being there when I needed him, for listening to me rant about many university projects, or crazy ideas. Furthermore, I would also like to thank my mother, for always being there for me, and for being the best mother I could have ever asked for, and for always supporting me in everything I do, even if we don't always agree on everything.

    I would not have been able to finish my degree without the help of my friends, who have always been there for me, adopting me as part of their family and being there when I needed them most. During the four years of my degree, I have met many people, some have persevered in completing their degree, while others have chosen to take a different path in life. I would like to thank all of them for being there for me, and for being part of my life. 

    Another person I would like to thank is my thesis advisor Jose Daniel. Thank you for giving me the opportunity to work with you, after an extreme late request for a thesis. I am grateful for the trust you placed in me from the very beginning and for guiding me when I was most lost. From the admiration, it has been a great pride to close this stage with you as my advisor.


  \end{acknowledgements}


  % ABSTRACT
  \begin{abstract}

  Nowadays, the importance of energy efficiency is increasing as more computationally expensive programs are being used by more and more people. The energy impact of running these programs is directly related with the programming language used to create the program as well as the design specifics. 

  This thesis aims to bring a specific example of the power efficiency of three programming languages: Python, Go and C++. Each one having its differences and properties, ease of use and execution speed. Each of these languages has been selected as each one has a particular characteristic that can be representative of their respective category of language.

  Compiled languages with no garbage collection and no managed runtime have usually had the best execution speed as they can reach byte-code for each specific platform, but in the last years, other methods have improved significantly, such as JIT (Just in Time) compiling

  To achieve realistic results, these languages were tested in multiple configurations, on different hardware, core count and operating systems to be able to eliminate any outliers. 

  Thus, this work will try showing the differences in energy consumption of different programming languages in a real world task, rendering a ray-traced image of multiple spheres with different materials and reflectivity.
  
    
    \keywords{Compiled Language \sep Energy Efficiency \sep Interpreted Language \sep JIT \sep Ray-Tracing}
  \end{abstract}


  % TOC
  \tableofcontents 
  \listoffigures
  \listoftables


  % THESIS
  \begin{thesis}
    \includefrom{parts/}{introduction.tex}
    \includefrom{parts/}{state_of_the_art.tex}
    \includefrom{parts/}{problem_statement}
    \includefrom{parts/}{design.tex}
    \includefrom{parts/}{evaluation.tex}
    \includefrom{parts/}{planification.tex}
    \includefrom{parts/}{economic.tex}
    \includefrom{parts/}{regulation.tex}
    \includefrom{parts/}{conclusions.tex}
    \newpage
    \import{parts/}{example.tex}
  \end{thesis}

  % BIBLIOGRAPHY
  \cleardoublepage
  \label{bibliography}
  \printbibliography[heading=bibintoc]


  % GLOSSARY
  \cleardoublepage
  \label{glossary}
  \printglossaries
  % \printnoidxglossaries[type=\acronymtype]  % slower, but no need to do $ makeglossaries report


  % APPENDICES
  % \begin{appendices}
  %   \chapter{My stuff}
  %   \lipsum
  % \end{appendices}


\end{document}
